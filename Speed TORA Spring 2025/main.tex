\documentclass{beamer}
\usepackage[utf8]{inputenc}
\usepackage{graphicx}
\usepackage{tikz-cd}
\usepackage{ulem}
\usepackage{multicol}

\usepackage{untgreen}
\UNTGpalette
\usetheme{Warsaw}
\setbeamertemplate{navigation symbols}{}%remove navigation symbols

\usepackage{xcolor}
\definecolor{lightgreen}{RGB}{32, 165, 32}
\definecolor{darkgreen}{RGB}{0,105,0}
\definecolor{lightred}{RGB}{209,27,27}
\definecolor{darkred}{RGB}{162,17,17}
\definecolor{darkblue}{RGB}{30,30,187}

\title{Deformations of Hopf-Ore Smash Products}
\author{Brandon Mather}
\date{TORA, April 2025}


\begin{document}

\maketitle

\begin{frame}{Hopf-Ore Smash Products}
\begin{definition}[Ore Extensions]
    For a \(k-\)alg \(R\), \textbf{Ore extension} \(R[x;\sigma,\delta]\) has product from \(k[x]\) with relation \(xr=\sigma(r)x+\delta(r)\) where \(\sigma\in \text{Aut}R\) and \(\delta\) a \(\sigma-\)derivation. 
\end{definition}
\begin{definition}
    If \(H\) is a Hopf algebra, \(H[x;\sigma,\delta]\) is a \textbf{Hopf-Ore Extension} if it is both an Ore extension and a Hopf Alg with \(R\) a sub-Hopf algebra, and \(\triangle(x)=x\otimes 1+g\otimes x\) some \(g\in G(R)\).
\end{definition}
\begin{definition}
    A filtered alg \(\mathcal{H}\) is a \textbf{PBW deformation} of its homogeneous version if it has the PBW property.
\end{definition}
\end{frame}

\begin{frame} 
    Let \(H\) be a Hopf alg, \(R\) a Koszul alg
\begin{beamerboxesrounded}{Question}
    Given a PBW deformation of \(R\# H\), under what conditions do we also get a PBW deformation of \(R\#H[x;\sigma,\delta]\)?
\end{beamerboxesrounded}

\underline{Techniques:}
\begin{itemize}
    \item Explicit PBW conditions for \(R\#H\) given by Shepler \& Witherspoon
    \item Hochschild cohomology techniques specific to twisted tensor products
\end{itemize}
\begin{example}
    Consider the Koszul alg \(R=\mathbb{C}[x]\) and the Hopf alg \(H=\mathbb{C}[y]\), a Hopf-Ore extension of \(\mathbb{C}\), \(H\) acts on \(R\) by \(y\cdot x=0\).
    \(R\#H\) is iso to \(\mathbb{C}[x][y;\text{id},\text{id}]\), which is not PBW, what could the PBW deformations be?
\end{example}
\end{frame}
\end{document}
