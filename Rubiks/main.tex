\documentclass{amsart}
\usepackage[utf8]{inputenc}
\newtheorem{theorem}{Theorem}
\theoremstyle{definition}
\newtheorem{definition}{Definition}
\DeclareMathOperator{\Sym}{Sym}
\title{Rubiks}
\author{Logan Crone}
\date{September 2021}

\begin{document}

\maketitle

\section{Introduction}

\begin{definition}
A \emph{2d-Rubik's shape} is a graph $G$, with a set of distinguished cycles $\{C_1, \dots, C_n\}$ of $G$  which is built up by induction as follows:

\begin{enumerate}
\item The $n$-cycles for $n \geq 3$ are all 2d-Rubik's shapes.
\item Given a 2d-Rubik's shape $G$ with distinguished cycles $\{C_1, \dots C_n\}$, if we let $P$ be a path in $G$ with at least two vertices so that $E(P) \cap E(C_i) \cap E(C_j) = \emptyset$ for any distinct $i, j$, and let $G' = G \cup \{v_1, \dots v_k\}$ with $v_i E' v_{i+1}$ and $v_1$ is connected to one endpoint of $P$ by an edge, and $v_k$ is connected to the other endpoint of $P$, then $G'$, together with the distinguished cycles $\{C_1, \dots C_n\} \cup \{P \cup \{v_1, \dots v_k\}\}$ is a 2d-Rubik's shape.
\end{enumerate}
\end{definition}

\begin{definition}
A 2d-Rubik's shape $G, \{C_1, \dots C_n\}$ is \emph{good} if the subgroup of $H \leq \Sym(E(G))$ generated by the permutations $\sigma_1, \dots, \sigma_n$ which respectively rotate the edges of the cycles $C_1, \dots, C_n$ is equal to $\Sym(E(G))$.
\end{definition}
The intuition behind this definition is that a good 2d-Rubik's shape can be returned by use of the allowed permutations $\sigma_1, \dots \sigma_n$ from any initial configuration of the edges to the starting configuration.

\begin{theorem}
Suppose that $G$ is a 2d-Rubik's shape with distinguished cycles ${C_1, C_2}$ and that the length of at least one of $C_1, C_2$ is even.  Then $G$ is a good 2d-Rubik's shape.
\end{theorem}
\begin{proof}
Suppose without loss of generality that $\sigma_1$ has even order.
Label the edges of $C_1$ by $(0, 1, \dots, k)$ and the edges of $C_2$ by $(m, m-1, \dots, 1, 0, k+\ell, k+\ell-1, \dots, k+2, k+1)$, so that if we identify $\sigma_1, \sigma_2$ with the corresponding elements of $\Sym(k+\ell+1)$ then $\sigma_1$ is the cycle $(0, 1, \dots, k)$ and $\sigma_2$ is the cycle $(m, m-1, \dots, 1, 0, k+\ell, k+\ell-1, \dots, k+2, k+1)$

We claim that if $0\leq j-1 < j \leq k$ then the product
\[\sigma_1^{i+j}(\sigma_2^{-m-1}\sigma_1^2\sigma_2^{m+1}\sigma_1)^{\frac{k-2}{2}}\sigma_2^{-1}\sigma_1^2\sigma_2\sigma_1^{k-j}\]
will transpose the edges $j-1$ and $j$.
\end{proof}


\end{document}
