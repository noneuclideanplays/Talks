\documentclass[12pt,a4paper]{article}
\usepackage[english]{babel}
\usepackage[utf8x]{inputenc}
\usepackage[T1]{fontenc}
\usepackage{listings}
\usepackage{amsfonts}
\usepackage{amssymb}
\usepackage{amsmath,amsthm}
\usepackage{mathtools,leftindex,tensor,mhchem}
\usepackage{graphicx}
\usepackage{geometry}
\usepackage{layout}
\usepackage{tikz}
\usepackage{bbm}
\usepackage{tikz-cd}
\usepackage{cite}
\usepackage{authblk}
\usetikzlibrary{positioning}
\usepackage{latexsym,amsxtra,amscd,ifthen, amsmath,color,url}
\newcommand{\comp}{\overline{\phantom{A}}}
\setlength{\tabcolsep}{12pt}

\DeclareMathOperator{\Hom}{Hom}
\newcommand{\Wedge}{\textstyle\bigwedge}

\newtheorem{theorem}{Theorem}[section]
\newtheorem{lemma}[theorem]{Lemma}
\newtheorem{claim}[theorem]{Claim}
\newtheorem{proposition}[theorem]{Proposition}
\newtheorem{corollary}[theorem]{Corollary}
\newtheorem{fact}[theorem]{Fact}
\newtheorem{example}[theorem]{Example}
\newtheorem{notation}[theorem]{Notation}
\newtheorem{observation}[theorem]{Observation}
\newtheorem{conjecture}[theorem]{Conjecture}
\newtheorem{remark}[theorem]{Remark}
\newtheorem{definition}[theorem]{Definition}
\newcommand\NN{\mathbb{N}} 
\newcommand\RR{\mathbb{R}}
\newcommand\CR{\mathcal{R}}
\newcommand\ZZ{\mathbb{Z}}
\newcommand\AL{\mathcal{A}}
\newcommand\MM{\mathcal{M}}
\newcommand\BB{\mathcal{B}}
\newcommand\BBB{\mathfrak{B}}
\newcommand\CC{\mathcal{C}}
\newcommand\CCC{\mathfrak{C}}
\newcommand\DD{\mathcal{D}}
\newcommand\EE{\mathbb{E}}
\newcommand\FF{\mathcal{F}}
\newcommand\GG{\mathcal{G}}
\newcommand\HH{\mathcal{H}}
\newcommand\JJ{\mathcal{J}}
\newcommand\PP{\mathcal{P}}
\newcommand\QQ{\mathbb{Q}}
\newcommand\CS{\mathcal{S}}
\newcommand\TT{\mathcal{T}}
\newcommand\BT{\mathbb{T}}
\newcommand\OO{\mathcal{O}}
\newcommand\UU{\mathcal{U}}
\newcommand\VV{\mathcal{V}}
\newcommand{\kk}{\Bbbk}
\newcommand\PPT{\mathcal{P}^{*}}
\newcommand\CST{\Sigma^{*}}
\newcommand\RX{\mathsf{X}}
\newcommand\1{_{(1)}}
\newcommand\2{_{(2)}}

%\voffset=-1truein		% LaTeX has too much space at page top
%\addtolength{\textheight}{0.3truein}
%\addtolength{\textheight}{\topmargin}
%\addtolength{\topmargin}{-\topmargin}
%\textwidth  6.0in		% LaTeX article default 360pt=4.98''
%\oddsidemargin 0pt	% \oddsidemargin  .35in   % default is 21.0 pt
%\evensidemargin 0pt	% \evensidemargin .35in   % default is 59.0 pt
%\marginparwidth=0pt
% decrease margins on sides and top/bottom
\geometry{
bottom=25mm,
top=35mm,
left=20mm,
right=20mm
}
\voffset=-0.5truein

\title{Koszul Resolution of a Skew Polynomial Ring}
\author{Brandon Mather}
\date{\today}
\affil{Departments of Mathematics, University of North Texas}

\begin{document}

\maketitle

We consider the skew polynomial ring $A=\kk_q[x_1,x_2]$ and construct its Koszul resolution. 
This allows the computation of its Hochschild cohomology groups.

\section{Koszul Resolution}

Let $\kk$ be a field, $A=\kk_q[x_1,x_2]=\kk<x_1,x_2>/(x_2x_1-qx_1x_2)$ for some $q\in\kk^*$.
Let $A_1=\kk[x_1]$ and $A_2=\kk[x_2]$ be subalgebras of $A$ so that $A$ is the twisted tensor product $A_1\otimes^\tau A_2$ where $\tau:\ZZ^2\to\kk^*$ is the bicharacter $\tau(m,n)= q^{-mn}$.
We consider the Koszul resolutions of $A_1$ and $A_2$
\[
  0\to A_1^e\xrightarrow{(x_1\otimes 1-1\otimes x_1)\cdot}A_1^e\to 0
\]
\[
0\to A_2^e\xrightarrow{(x_2\otimes 1-1\otimes x_2)\cdot}A_2^e\to 0.  
\]
In order to construct a resolution of $A$,, the differentials of these resolutions need to be graded maps.
To this end, we shift the grading of the homological degree 1 component of both resolutions up by 1:
\begin{align*}
0\to A_1^e(-1)\xrightarrow{(x_1\otimes 1-1\otimes x_1)\cdot}A_1^e\to 0\\
0\to A_2^e(-1)\xrightarrow{(x_2\otimes 1-1\otimes x_2)\cdot}A_2^e\to 0.  
\end{align*}
Then, the differential $(x_1\otimes 1-1\otimes x_1)\cdot$ maps basis elements as follows
\[
(x_1\otimes 1-1\otimes x_1)(x_1^n\otimes x_1^m)=x_1^{n+1}\otimes x_1^m-x_1^n\otimes x_1^{m+1}.
\]
The element on the right has degree $n+m+1$ in $A_1^e$ and the element $x_1^n\otimes x_1^m$ has shifted degree $n+m+1$ in $A_1^e(-1)$, so we see the differential is now graded.
Similarly, the differential $(x_2\otimes 1-1\otimes x_2)\cdot:A_2^e(-1)\to A_2^e$ is graded.
\\

Then by a theorem proved by Bergh and Oppermann in "Cohomology of Twisted Tensor Products" (2008), the total complex of the tensor product of these two resolutions is a projective resolution of $A$ as an $A^e$-module.
This resolution is 
\[
0\to A_1^e(-1)\otimes A_2^e(-1)\xrightarrow{\partial_2}\left[A_1^e\otimes A_2^e(-1)\right]\oplus \left[A_1^e(-1)\otimes A_2^e\right]\xrightarrow{\partial_1}A_1^e\otimes A_2^e\to 0
\]
where the differentials are given by
\[
\partial_2=\begin{bmatrix}(x_1\otimes 1-1\otimes x_1)\cdot\otimes \text{id}\\\text{id}\otimes(1\otimes x_2-x_2\otimes 1)\cdot\end{bmatrix}:x_1^a\otimes x_1^b\otimes x_2^c\otimes x_2^d\mapsto\begin{bmatrix}x_1^{a+1}\otimes x_1^b\otimes x_2^c\otimes x_2^d-x_1^a\otimes x_1^{b+1}\otimes x_2^c\otimes x_2^d\\ x_1^a\otimes x_1^b\otimes x_2^c\otimes x_2^{d+1}-x_1^a\otimes x_1^b\otimes x_2^{c+1}\otimes x_2^d\end{bmatrix}
\]

\newpage
\[\partial_1=\begin{bmatrix}\text{id}\otimes(x_2\otimes 1-1\otimes x_2)\cdot&(x_1\otimes 1-1\otimes x_1)\cdot\otimes\text{id}\end{bmatrix}:\]\\
\[\begin{bmatrix}x_1^a\otimes x_1^b\otimes x_2^c\otimes x_2^d\\x_1^r\otimes x_1^s\otimes x_2^u\otimes x_2^v\end{bmatrix}\mapsto x_1^a\otimes x_1^b\otimes x_2^{c+1}\otimes x_2^d-x_1^a\otimes x_1^b\otimes x_2^c\otimes x_2^{d+1}
+ x_1^{r+1}\otimes x_1^s\otimes x_2^u\otimes x_2^v-x_1^r\otimes x_1^{s+1}\otimes x_2^u\otimes x_2^v.
\]

Of note is that the last module, $A_1^e\otimes A_2^e$, is isomorphic to $A^e$ as $A^e$-modules via the map
\[
x_1^a\otimes x_1^b\otimes x_2^c\otimes x_2^d\mapsto q^{-bc} x_1^ax_2^c\otimes x_1^bx_2^d.  
\]
For the sake of easier computation of the cohomology, we will perform similar isomorphisms of the other two $A^e$-modules,
\begin{align*}
A^e_1(-1)\otimes A^e_2(-1)&\cong A^e\\
x_1^a\otimes x_1^b\otimes x_2^c\otimes x_2^d&\mapsto q^{-(b+1)(c+1)}x_1^ax_2^c\otimes x_1^bx_2^d
\end{align*}
and
\begin{align*}
\left(A^e_1\otimes A^e_2(-1)\right)\oplus\left(A^e_1(-1)\otimes A^e_2\right)&\cong A^e\oplus A^e\\
\begin{bmatrix} x_1^a\otimes x_1^b\otimes x_2^c\otimes x_2^d\\x_1^r\otimes x_1^s\otimes x_2^u\otimes x_2^v\end{bmatrix}&\mapsto\begin{bmatrix}q^{-b(c+1)}x_1^ax_2^c\otimes x_1^bx_2^d\\q^{-(s+1)u}x_1^rx_2^u\otimes x_1^sx_2^v \end{bmatrix}
\end{align*}

Hence, we can rewrite the resolution as
\[
0\to A^e\xrightarrow{\partial^*_2}A^e\oplus A^e\xrightarrow{\partial^*_1}A^e\to 0
\]
where the differentials are given by
\begin{align*}
\partial^*_2\left(x_1^ax_2^b\otimes x_1^cx_2^d\right)&=\begin{bmatrix}q^{b+1}x_1^{a+1}x_2^b\otimes x_1^cx_2^d-x_1^ax_2^b\otimes x_1^{c+1}x_2^d\\q^{c+1}x_1^ax_2^b\otimes x_1^cx_2^{d+1}-x_1^ax_2^{b+1}\otimes x_1^cx_2^d \end{bmatrix}\\
\partial^*_1\left(\begin{bmatrix}x_1^ax_2^b\otimes x_1^cx_2^d\\ x_1^rx_2^s\otimes x_1^ux_2^v\end{bmatrix}\right)&=x_1^ax_2^{b+1}\otimes x_1^cx_2^d-q^{c}x_1^ax_2^b\otimes x_1^cx_2^{d+1}+q^{s}x_1^{r+1}x_2^s\otimes x_1^ux_2^v-x_1^rx_2^s\otimes x_1^{u+1}x_2^v.
\end{align*}

\newpage

\section{Computing Cohomology}
Apply the functor $\Hom_{A^e}(-,A)$ to this resolution to get the complex
\[
0\to \Hom_{A^e}(A^e,A)\xrightarrow{d_1} \Hom_{A^e}(A^e\oplus A^e,A)\xrightarrow{d_2}\Hom_{A^e}(A^e,A)\to 0.  
\]
The differentials are given by $d_i(f)(a)=f(\partial_i^*(a))$.

Next, we want to rewrite this complex in a more familiar form.
Let $V=\kk x_1\oplus \kk x_2$, then
\begin{align*}
\Hom_{A^e}(A^e,A)&\cong A\otimes \Wedge^2_q (V)\\
f&\mapsto f(1\otimes 1)\otimes x_1\wedge x_2
\end{align*}
This map is clearly $A^e$-linear and surjective.
Since every $f\in \Hom_{A^e}(A^e,A)$ is determined by its image on $1\otimes 1$, it is also injective, and hence, an isomorphism of $A^e$-modules.
\\

We also have that
\begin{align*}
\Hom_{A^e}(A^e\oplus A^e,A)&\cong A\otimes \Wedge_q^1(V)\\
f&\mapsto f(1\otimes 1,0)\otimes x_1+f(0,1\otimes 1)\otimes x_2
\end{align*}
and
\begin{align*}
\Hom_{A^e}(A^e,A)&\cong A\otimes \Wedge_q^0 (V)\\
f&\mapsto f(1\otimes 1).
\end{align*}

Ultimately, we have the complex
\[
0 \to A\otimes \Wedge_q^2(V)\xrightarrow{d_1^*}A\otimes \Wedge_q^1(V)\xrightarrow{d_2^*}A\otimes \Wedge_q^0(V)\to 0
\]
with differentials given by
\begin{align*}
d_1^*(x_1^ax_2^b\otimes x_1\wedge x_2)&=(q^{a}-q)x_1^ax_2^{b+1}\otimes x_1+(q-q^{b})x_1^{a+1}x_2^b\otimes x_2\\
d_2^*(x_1^ax_2^b\otimes x_1+x_1^cx_2^d\otimes x_2)&=(q-q^{b})x_1^{a+1}x_2^b+(q-q^{c})x_1^cx_2^{d+1}
\end{align*}

\end{document}