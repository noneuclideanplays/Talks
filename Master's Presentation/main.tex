\documentclass{beamer}
\usepackage[utf8]{inputenc}
\usepackage{graphicx}
\usepackage{tikz-cd}
\usepackage{ulem}
\usepackage{multicol}

\usepackage{untgreen}
\UNTGpalette
\usetheme{Warsaw}
\setbeamertemplate{navigation symbols}{}%remove navigation symbols

\usepackage{xcolor}
\definecolor{lightgreen}{RGB}{32, 165, 32}
\definecolor{darkgreen}{RGB}{0,105,0}
\definecolor{lightred}{RGB}{209,27,27}
\definecolor{darkred}{RGB}{162,17,17}
\definecolor{darkblue}{RGB}{30,30,187}

\title{Quantum Symmetry of Hopf Actions}
\author{Brandon Mather}
\date{Algebra Seminar, November 2023}


\begin{document}

\maketitle

\begin{frame}{History}
    \begin{itemize}
        \item (1939) Heinz Hopf works on homology of sphere groups
        \item (1969) Moss Sweedler writes seminal book "Hopf Algebras"
        \item (1986) Vladimir Drinfeld gives ICM address on quantum groups
        \item (1992) Susan Montogomery writes "Hopf Algebras and Their Actions on Rings"
    \end{itemize}
    
\end{frame}

\begin{frame}{Kac-Paljutkin Algebra}

 The unique $8$-dim'l non-commutative, non-cocommutative Hopf alg given by G.\ Kac and V.\ Paljutkin in``Finite Ring Groups" (1966):\newline

    $H_8=$
    \[\Big\langle x,y,z\;\vert\; x^2=y^2=1,xy=yx,zx=yz,zy=xz,z^2=\tfrac{1}{2}(1+x+y-xy)\Big\rangle\]
    with operations
    \vspace{2ex}
    \begin{itemize}
    \setlength{\itemsep}{2ex}
    \item[]<2->
    $\triangle(x)=x\otimes x,\; \triangle(y)=y\otimes y,\;$ 
    
    \vspace{2ex}
    
    $\triangle(z)=\tfrac{1}{2}(1\otimes 1+1\otimes x+y\otimes 1-y\otimes x)(z\otimes z),\;$
    
    \item[]<3->$
    \varepsilon(x)=\varepsilon(y)=\varepsilon(z)=1,$

    \item[]<4->$ S(x)=x,\;S(y)=y,\;S(z)=z.$ 
    \end{itemize}
\end{frame}

\begin{frame}{Actions of Kac-Paljutkin Algebra}

\begin{itemize}
    \item[]<1->$H_8$ acts on $\mathbb{C}_q[u,v]$ where $q^2=-1$ by 
    \[
    x\mapsto \begin{bmatrix}-1 & 0\\0 & 1\end{bmatrix} \;\; y\mapsto \begin{bmatrix} 1 & 0\\ 0 & -1\end{bmatrix} \;\; z\mapsto\begin{bmatrix} 0 & 1\\ 1 & 0\end{bmatrix}.
    \]
    \item[]<2-> \hrule
    \item[]<2->And on $\mathbb{C}_Q[v_1,v_2,v_3,v_4]$ for 
   
    \vspace{1ex}
    $q_{12}=q_{34}^{-1},\;\;q_{13}=q_{24}^{-1},\;\;q_{14}^2=1,\;\;q_{23}^2=-1$ by

%Try 
%$\left[
%\begin{smallmatrix} 1&0&0&0\\0&-1&0&0\\0&0&1&0\\0&0&0&1\end{smallmatrix}
%\right]$

    
    \[
    x\mapsto \begin{bmatrix} 1&0&0&0\\0&-1&0&0\\0&0&1&0\\0&0&0&1\end{bmatrix} \;\; y\mapsto \begin{bmatrix} 1&0&0&0\\0&1&0&0\\0&0&-1&0\\0&0&0&1\end{bmatrix}\;\; z\mapsto\begin{bmatrix}0&0&0&1\\0&0&1&0\\0&1&0&0\\1&0&0&0\end{bmatrix}.
    \]
    
    \item[]<3->\hrule
    \item[]<3-> And on $\mathbb{C}_{-1}[u,v]$ by 
    \[
    x\mapsto \begin{bmatrix}0&1\\1&0\end{bmatrix}\;\; y\mapsto\begin{bmatrix}0&-1\\-1&0\end{bmatrix}\;\; z\mapsto\begin{bmatrix}1&0\\0&-1\end{bmatrix}.
    \]
\end{itemize}
\end{frame}

\begin{frame}{Quantized Universal Enveloping Algebra}

Described by Piotr Kulish and Nicolai Reshetikhin in``Quantum linear problem for the sine-Gordon equation and highest weight representations" (1983), leading Vladimir Drinfeld to quantum groups\newline

$\mathcal{U}_q(\mathfrak{sl}
_2)=$
\[
\Big\langle E,F,K,K^{-1}\;\vert\;EF-FE=(q-q^{-1})^{-1}\left(K-K^{-1}\right), KEK^{-1}=q^2 E, \]\[\hspace{4ex}KFK^{-1}=q^{-2}F, KK^{-1}=K^{-1}K=1\Big\rangle
\]
with operations:
\vspace{0.5ex}
\begin{itemize}
\setlength{\itemsep}{1.5ex}
    \item[]<2-> $\triangle(E)=E\otimes 1+K\otimes E,\; \triangle(F)=F\otimes K^{-1}+ 1\otimes F$, \item[]<2-> $\triangle(K)=K\otimes K,\; \triangle(K^{-1})=K^{-1}\otimes K^{-1}$,
    \item[]<3-> $\varepsilon(E)=\varepsilon(F)=0,\; \varepsilon(K)=\varepsilon(K^{-1})=1$,
    \item[]<4-> $S(E)=-K^{-1}E,\; S(F)=-F K, \;S(K)=K^{-1},\; S(K^{-1})=K$.
\end{itemize}

\end{frame}

\begin{frame}{Actions of $\mathcal{U}_q(\mathfrak{sl}_2)$}

$\mathcal{U}_q(\mathfrak{sl}_2)$ acts on $\mathbb{C}_q[u,v]$ by the representation
\begin{itemize}
\item[]<2->\[
E\mapsto \begin{bmatrix} 0&1\\0&0\end{bmatrix}\;\;\;\; F\mapsto \begin{bmatrix} 1&0\\0&0\end{bmatrix}
\]
\item[]<3->\[
K\mapsto\begin{bmatrix}q&0\\0&q^{-1}\end{bmatrix} \;\;\;\; K^{-1}\mapsto \begin{bmatrix}q^{-1}&0\\0&q\end{bmatrix}
\]
\end{itemize}

\end{frame}

\begin{frame}[fragile]{Hopf Algebra}

A \textbf{Hopf algebra} is a bialgebra $H$ over a field with an antipode \\\;\;\;\;\;\;\;\;$\color{darkblue}S:H\to H$ \;\;\;where the bialgebra operations are 
\vspace{-2ex}
\begin{multicols}{2}
\begin{itemize}
\item[]<2->$\color{lightred}\bigtriangledown:H\otimes H\to H$, 
\item[]<3->$\color{lightgreen}\;\eta:\mathbb{C}\to H$, 
\item[]<4->$\color{darkred}\triangle:H\to H\otimes H$, 
\item[]<5->$\color{darkgreen}\;\varepsilon:H\to\mathbb{C}$
\end{itemize}
\end{multicols}
\vspace{-2ex}
\pause so that the following commute:

\begin{minipage}[t]{0.3\textwidth}
\begin{itemize}
    \item[]<2->\text{Associativity:}
\begin{tikzcd}[color=lightred,cramped]
H\otimes H\otimes H \arrow[d, "id\otimes \bigtriangledown"'] \arrow[r, "\bigtriangledown\otimes id"] & H\otimes H \arrow[d, "\bigtriangledown"] \\
H\otimes H \arrow[r, "\bigtriangledown"']                                                         & H                                    
\end{tikzcd}
\item[]<3->\text{Unit:}
% https://tikzcd.yichuanshen.de/#N4Igdg9gJgpgziAXAbVABwnAlgFyxMJZARgBoAGAXVJADcBDAGwFcYkQBBAHS4jwFt4AAg4gAvqXSZc+QinKli1Ok1bseAaw08+WQXBHjJIDNjwEiZAEzKGLNok5GpZ2USuLbqh5x0DhmhriyjBQAObwRKAAZgBOEPxIZCA4EEgKKvbqXDA49H56wlhQziBxCUgAzDSp6TR2ao7FBfo8ufQgNIz0AEYwjAAK0uZyILFYYQAWOKXliYjJtYgemY0gALydIN19g8NujuNTMxIx8fPVKWnL9d7sm6dl53VXSCsNPjw44-RgYYyhCAAd0IYkoYiAA
\begin{tikzcd}[cramped,sep=small,color=lightgreen]
& H\otimes H \arrow[dd, "\bigtriangledown"] &                                                         \\
\mathbb{C}\otimes H \arrow[ru, "\eta\otimes id"] \arrow[rd, "="'] &                    & H\otimes \mathbb{C} \arrow[lu, "id\otimes\eta"'] \arrow[ld, "="] 
                \\
& H &                                                    
\end{tikzcd}
\end{itemize}
\end{minipage}~\hspace{1.5cm}\begin{minipage}[t]{0.48\linewidth}
    \begin{itemize}
        \item[] <4->\text{Coassociativity:} % https://tikzcd.yichuanshen.de/#N4Igdg9gJgpgziAXAbVABwnAlgFyxMJZABgBpiBdUkANwEMAbAVxiRAAkQBfU9TXfIRQBGclVqMWbdgB0ZEPAFt4AAk48+2PASJlh4+s1aIOchVmVw13XiAxbBRUfuqGpJ2fKWrP5y9a5xGCgAc3giUAAzACcIRSQyEBwIJFEJIzY5HGisOjAQhlYNEBi4hOpkpAAmV0ljECycvIKi21L4xBqklMQAZlqMk0bc-MKQajgACyxInATi9tSKnv709waZbJGWs28rLChuCi4gA
\begin{tikzcd}[color=darkred,cramped]%[sep=small]
H \arrow[r, "\triangle"] \arrow[d, "\triangle"'] & H\otimes H \arrow[d, "\triangle\otimes id"] \\
H\otimes H \arrow[r, "id\otimes\triangle"']               & H\otimes H\otimes H        
\end{tikzcd}
        \item[]<5->\text{Counit:}
% https://tikzcd.yichuanshen.de/#N4Igdg9gJgpgziAXAbVABwnAlgFyxMJZATikZ package - Overleaf, Online LaTeX EditorBgBoBGAXVJADcBDAGwFcYkQAdDga264jwBbeAAIAEiAC+pdJlz5CKcqWLU6TVuwnTZ2PASIAmCmoYs2iEGP5D4XXlJkgMehUWWHTGi1ZtZhcOJSajBQAObwRKAAZgBOEIJIAMw0OBBIZOrm7FwMsTBo2IwEfgEiWFCOMfGJiMogaRk0ZpqWALwgNIz0AEYwjAAKcvqKILFYYQAWOFUgcQlI9Y2IxlmtIB06czXJqekrzd7sFaV2HHkFRQZdvf1DrgaW41MzW-O1S-spaz5cOOP0MBhRhsSSUSRAA
\begin{tikzcd}[cramped,sep=small,color=darkgreen]
             & H \arrow[ld, "="'] \arrow[rd, "="] \arrow[dd, "\triangle"]                         &             \\
\mathbb{C}\otimes H &                                                                                    & H\otimes\mathbb{C} \\
             & H\otimes H \arrow[lu, "\varepsilon\otimes id"] \arrow[ru, "id\otimes\varepsilon"']&            
\end{tikzcd}
    \end{itemize}
\end{minipage}

\end{frame}

\begin{frame}[fragile]{Hopf Algebra Diagrams}

\begin{itemize}
\item[]<1->\text{Product and Coproduct compatibility:}
% https://tikzcd.yichuanshen.de/#N4Igdg9gJgpgziAXAbVABwnAlgFyxMJZABgBpiBdUkANwEMAbAVxiRAAkAdTiPAW3gACdiAC+pdJlz5CKAIzkqtRizYjxk7HgJEATIur1mrRB268sAuMLESQGLTKJk5So6tNce-IV4tXhcx9rdTsHaR0UfVdDFRMzb0tfIKSQlID1JRgoAHN4IlAAMwAnCD4kMhAcCCQAZljjNm4cYqw6MByGGHT4Ztb2ztYNEBKyuupqpAAWBo8QLCgeuGa6JiXBBdsi0vLESsnEBWVG0z62jq6oCAB3QmHR3aOD-WO5s4GurZGd6YmaxBe7ni7wu2RuYCWIMGV1uYgooiAA
\begin{tikzcd}[color=lightred]
H\otimes H \arrow[d, "\triangle\otimes\triangle"'] \arrow[r, "\bigtriangledown"] & H \arrow[r, "\triangle"] & H\otimes H                                                                  \\
H\otimes H\otimes H\otimes H \arrow[rr, "id\otimes\tau\otimes id"']           &                          & H\otimes H\otimes H\otimes H \arrow[u, "\bigtriangledown\otimes\bigtriangledown"']
\end{tikzcd}
\item[]<2->\text{Unit and Counit compatibility:}
% https://tikzcd.yichuanshen.de/#N4Igdg9gJgpgziAXAbVABwnAlgFyxMJZABgBpiBdUkANwEMAbAVxiRAB12BbOnACwBGA4AGEAviDGl0mXPkIoAjOSq1GLNgAlJ0kBmx4CRZYtX1mrRCE2cIeLvAAE2qTIPyiAJhXVzGqy66+nJGKN6mvuqW1rb2ToFuIQrIAMw+ahZsnDz8QqISYqowUADm8ESgAGYAThBcSGQgOBBIyhn+HOwwOHQ6VbX1iI3NSN4gDHQCMAwACrKGCiDVWCV8OCCRmVac3XSxWA5wOz19IDV1rdQjiGN+0Zw4y3RgJQys1HB8WJXriG0TU1m8w8VjeP1O50GABYri1EGl2vd2I8sM9XqxXGcBkgEdcAKybDqcejVGBobAMIyYyFIGFNOEExFZdgkskUozUAHTObuUJLFZrSQUMRAA
%\begin{center}
\begin{tikzcd}[color=darkgreen]
\mathbb{C} \arrow[r, "\eta"] \arrow[rd, "\eta\otimes\eta"'] & H \arrow[d, "\triangle"] & \phantom{a} & H \arrow[r, "\varepsilon"]                                   & \mathbb{C} \\
                                                            & H\otimes H                            & \phantom{a} & H\otimes H \arrow[u, "\bigtriangledown"] \arrow[ru, "\varepsilon\otimes\varepsilon"'] &           
\end{tikzcd}
%\end{center}
\item[]<3->\text{Antipode:}
% https://tikzcd.yichuanshen.de/#N4Igdg9gJgpgziAXAbVABwnAlgFyxMJZABgBoBGAXVJADcBDAGwFcYkQAJEAX1PU1z5CKchWp0mrdgB1pAazk8+IDNjwEiAJjE0GLNok5L+aoUTKbxeqYY6yIeALbwABF14nBGlNsu7JBpz2Tq7uyqpewiSkxFYB7HbSDljOcG7GKgLqUdqx-voJwSmhPOIwUADm8ESgAGYAThCOSGQgOBBIAKz5NiCyOPVY9GAVjGweIA1NXTTtSABsPYFYUEWpLgDKGVPNiIttHYjaEgWG-YPDo+UQAO6EEzsts4cAzEsy0gNDI2MgNIz0ABGMEYAAUsmZDIMKgALHDbRq7N4HJAAFnehg2a1cKwR00Q6JRRwxfU+Fx+1zueN2rTmiFEJ16sgY9RgaGwjA0D0RSAZdOO1kCshgOHopW4QA
\begin{tikzcd}[color=darkblue]
H\otimes H \arrow[rr, "id\otimes S"]                                      &                       & H\otimes H \arrow[d, "\bigtriangledown"] \\
H \arrow[u, "\triangle"] \arrow[d, "\triangle"'] \arrow[r, "\varepsilon"] & \mathbb{C} \arrow[r, "\eta"] & H                                     \\
H\otimes H \arrow[rr, "S\otimes id"]                                      &                       & H\otimes H \arrow[u, "\bigtriangledown"']
\end{tikzcd}
\end{itemize}
\end{frame}

\begin{frame}[fragile]{Hopf Algebra Actions}

Let $H$ be a Hopf alg and $A$ an alg with a map $\alpha:H\otimes A\to A$. Then we say $H$ \textbf{acts} on $A$ by $\alpha$ if the following the diagrams commute:

\begin{itemize}
    \item[]<2->\begin{tikzcd}[color=lightred]
H\otimes H\otimes A \arrow[r, "\bigtriangledown\otimes id"] \arrow[d, "id\otimes \alpha"] & H\otimes A \arrow[d, "\alpha"] \\
H\otimes A \arrow[r, "\alpha"]                                                         & A                             
\end{tikzcd}
% https://tikzcd.yichuanshen.de/#N4Igdg9gJgpgziAXAbAABwnAlgFyxMJZABgBpiBdUkANwEMAbAAxiRAB12BrLziPALbwABADUQAX1LpMufIRQBGclAqMWbABJ9BI8AJnY8BIssWr6zAohD7AMKAHN4RUADMAThAFIyIHBBIympWbJwwOHQ6WEJwwlhQktIgnt6+1AFIAEzUlho2ALwg1Ax0AEYwDAAKssYKIB5YjgAWOEnuXj6IwZmIOSH5HOyMaM10khQSQA  
        \begin{tikzcd}[color=darkgreen]
\mathbb{C}\otimes A \arrow[r, "\eta\otimes id"] \arrow[rd, "="'] & H\otimes A \arrow[d, "\alpha"] \\
                                                          & A                                                                       
\end{tikzcd}
\vspace{1.5ex}
\end{itemize}
\pause \pause $A$ is called a \textbf{module algebra} if the following also commute:
\begin{itemize}
\item[]<4-> \begin{tikzcd}[color=darkblue]
H\otimes A\otimes A \arrow[r, "\bigtriangledown"] \arrow[d, "\triangle\otimes id\otimes id"'] & H\otimes A \arrow[r, "\alpha"] & A & A\otimes A \arrow[l, "\bigtriangledown"']                         \\
H\otimes H\otimes A\otimes A \arrow[rrr, "id\otimes\tau\otimes id"]                        &                                &   & H\otimes A\otimes H\otimes A \arrow[u, "\alpha\otimes\alpha"']
\end{tikzcd}.\
% https://tikzcd.yichuanshen.de/#N4Igdg9gJgpgziAXAbVABwnAlgFyxMJZABgBpiBdUkANwEMAbAVxiRAAkQBfU9TXfIRQBGclVqMWbADrSA1nO68QGbHgJEATGOr1mrRCACCSvmsFEyw8XqmH2siHgC28AAQmu4mFADm8IlAAMwAnCGckMhAcCCRRCX0ZaXoQmDRsBg0eYLCIxHiYpG0EuxBZGBw6UxBQ8MjqQsQAZl1JAzLpCqrsmtykFujYxGLbdtlGNAALboouIA
    \item[]<5->
     \begin{center}\begin{tikzcd}[color=darkgreen]
H \arrow[r, "\varepsilon"] \arrow[d, "\eta"'] & \mathbb{C} \arrow[r, "\eta"] & A \\
H\otimes A \arrow[rru, "\alpha"']             &                       &  
\end{tikzcd}\end{center}
\end{itemize}

\end{frame}

\end{document}
