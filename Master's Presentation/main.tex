\documentclass{beamer}
\usepackage[utf8]{inputenc}
\usepackage{graphicx}
\usepackage{tikz-cd}
\usepackage{ulem}
\usepackage{multicol}
\usepackage{comment}
\usepackage{mathtools}
\usepackage{untgreen}
\UNTGpalette
\usetheme{Warsaw}
\setbeamertemplate{navigation symbols}{}%remove navigation symbols

\usepackage{xcolor}
\definecolor{lightgreen}{RGB}{32, 165, 32}
\definecolor{darkgreen}{RGB}{0,105,0}
\definecolor{lightred}{RGB}{209,27,27}
\definecolor{darkred}{RGB}{162,17,17}
\definecolor{darkblue}{RGB}{30,30,187}

\title{Hopf Module Algebras}
\author{Brandon Mather}
\date{UNT Master's}


\begin{document}

\maketitle

\begin{frame}{History}
    \begin{itemize}\setlength{\itemindent}{-2ex}
        \item (1939) Heinz Hopf works on homology of sphere groups
        \item (1969) Moss Sweedler writes seminal book ``Hopf Algebras"
        \item (1986) Vladimir Drinfeld gives ICM address on quantum groups
        \item (1992) Susan Montogomery writes ``Hopf Algebras and Their Actions on Rings"
    \end{itemize}
    \begin{block}{Goal}
        To understand the actions of Hopf algebras on other algebras
    \end{block}
\end{frame}

\begin{frame}{Quantum Plane}
    Notation: $\mathbb{C}[v_1,\ldots,v_n]=\mathbb{C}\left\langle v_1,\ldots,v_n\;\vert\; v_jv_i-v_iv_j\right\rangle$
    \begin{block}{Quantum Polynomial Ring}
        Let $Q=(q_{ij})$ be an $n\times n$ matrix of roots of unity where \[q_{ii}=1=q_{ji}q_{ij}.\]
        \\
        A \textbf{quantum polynomial ring} is \[\mathbb{C}_Q[v_1,\ldots,v_n]=\mathbb{C}\left\langle v_1,\ldots,v_n \;\vert\; v_jv_i-q_{ij}v_iv_j\right\rangle.\]
    \end{block}
    Example: $\mathbb{C}_{-1}[v_1,v_2]=\mathbb{C}\left\langle v_1,v_2\;\vert\; v_1 v_2+v_2 v_1\right\rangle$
\end{frame}

\begin{frame}{Motivation}
\begin{columns}[T]
\begin{column}{0.5\textwidth}
\begin{enumerate}\setlength{\itemindent}{-2ex}
    \item<1-> When a grp $G$ acts on a space $V$ over $\mathbb{C}$ linearly, the action can be extended to $V\otimes V$ by $g\in G$ acting as \[g\otimes g=\triangle(g).\] 
     Then $\triangle$ defines a coproduct map \[\triangle:\mathbb{C}G\to\mathbb{C}G\otimes\mathbb{C}G.\]
     For arbitrary coproducts, $\triangle:A\to A\otimes A$, \\we call $g\in A$ \textbf{grouplike} if $\triangle(g)=g\otimes g$.
\end{enumerate}
\end{column}
\begin{column}{0.5\textwidth}
\begin{enumerate}
     \item<2-> When a Lie alg $\mathfrak{g}$ acts on a space $V$ over $\mathbb{C}$, the action can be extended to $V\otimes V$ by $x\in\mathfrak{g}$ acting as \[x\otimes 1+1\otimes x=\triangle(x).\]
      Then $\triangle$ defines a coproduct map \[\triangle:\mathfrak{g}\to\mathfrak{g}\otimes\mathfrak{g}.\]
      For arbitrary coproducts, $\triangle:A\to A\otimes A$,\\ we call $x\in A$ \textbf{primitive} if $\triangle(x)=x\otimes 1+1\otimes x$.
\end{enumerate}
\end{column}
\end{columns}
\vspace{0.5cm}
\end{frame}

\begin{frame}{}
    Hopf algebras combine the notions of Algebras and Coalgebras that have coproducts.
    We are looking for actions of Hopf algebras.
    In particular on non-commutative algebras like Quantum Polynomial Rings.
\end{frame}

\begin{frame}{Sweedler's Algebra}

%From: Walton, Witherspoon "Poincare-Birkhoff-Witt deformations.." 2014

    The unique $4$-dim'l non-commutative, non-cocommutative Hopf alg given by M. Sweedler (1969):\\
    \[
        H_4=\Big\langle g,x\;\vert\; g^2=1,x^2=0,gx=-xg\Big\rangle  
    \]
    \onslide<2->{with operations $\triangle:H_4\to H_4\otimes H_4,\varepsilon:H_4\to\mathbb{C},S:H_4\to H_4$}
\begin{align*}
    \onslide<2->{\triangle(g) & =g\otimes g, & \triangle(x)= & \ x\otimes 1+g\otimes x \\}
    \onslide<3->{\\}
    \onslide<3->{\varepsilon(g) & =1, & \varepsilon(x)= &\ 0\\}
    \onslide<4->{\\}
    \onslide<4->{S(g) & =g^{-1}, & S(x)= & \ -x\\}
    \onslide<5->{\rule{2cm}{0.4pt} & \rule{2cm}{0.4pt} & \rule{2cm}{0.4pt} & \rule{3cm}{0.4pt}\\}
    \onslide<5->{\text{Group-like} & & & (1,g)-\text{Primitive}}
\end{align*}
\onslide<6->{Let $\tau$ be the `flip' over the tensor product, so $\tau(u\otimes v)=v\otimes u$.\\
Note that $\tau\circ\triangle(x)\neq\triangle(x)$, so $H$ is non-cocommutative.}
\end{frame}

\begin{frame}{Actions of Sweedler's Algebra}
    $H_4$ acts on $\mathbb{C}_{-1}[v_1,v_2]$ by 
    \[
      g\cdot v_1=v_1,\; g\cdot v_2=-v_2,\; x\cdot v_1=0,\; x\cdot v_2=v_1.  
    \]
    \pause
   We can express this action on the generators as
    \[
    g\mapsto \begin{bmatrix}1&0\\0&-1\end{bmatrix},\; x\mapsto\begin{bmatrix}0&1\\0&0\end{bmatrix}.    
    \]
\end{frame}


\begin{frame}{Kac-Paljutkin Algebra}

%From: Kirkman, Kuzmanovich, Zhang, "Gorenstein subrings of invariants..." 2009

 The unique $8$-dim'l non-comm., non-cocomm. Hopf alg given by G.\ Kac and V.\ Paljutkin in``Finite Ring Groups" (1966):\newline

    $H_8=$
    \[\Big\langle x,y,z\;\vert\; x^2=y^2=1,xy=yx,zx=yz,zy=xz,z^2=\tfrac{1}{2}(1+x+y-xy)\Big\rangle\]
    with operations
    \vspace{2ex}
    \begin{itemize}
    \setlength{\itemsep}{2ex}
    \item[]<2->
    $\triangle(x)=x\otimes x,\; \triangle(y)=y\otimes y,\;$ 
    
    \vspace{2ex}
    
    $\triangle(z)=\tfrac{1}{2}(1\otimes 1+1\otimes x+y\otimes 1-y\otimes x)(z\otimes z),\;$
    
    \item[]<3->$
    \varepsilon(x)=\varepsilon(y)=\varepsilon(z)=1,$

    \item[]<4->$ S(x)=x,\;S(y)=y,\;S(z)=z.$ 
    \end{itemize}
\end{frame}

\begin{frame}{Actions of Kac-Paljutkin Algebra}

\begin{itemize}

%Need some way to make it clear x acts as matrix etc.

    \item[]<1->$H_8$ acts on $\mathbb{C}_q[v_1,v_2]$ where $q^2=-1$ via the representation
    \[
    x\mapsto \begin{bmatrix}-1 & 0\\0 & 1\end{bmatrix} \;\; y\mapsto \begin{bmatrix} 1 & 0\\ 0 & -1\end{bmatrix} \;\; z\mapsto\begin{bmatrix} 0 & 1\\ 1 & 0\end{bmatrix}.
    \]
    \item[]<2-> \hrule
    \item[]<2->And on $\mathbb{C}_Q[v_1,v_2,v_3,v_4]$ for 
   
    \vspace{1ex}
    $q_{12}=q_{34}^{-1},\;\;q_{13}=q_{24}^{-1},\;\;q_{14}^2=1,\;\;q_{23}^2=-1$ via the rep
    
    \[
    x\mapsto \begin{bmatrix} 1&0&0&0\\0&-1&0&0\\0&0&1&0\\0&0&0&1\end{bmatrix} \;\; y\mapsto \begin{bmatrix} 1&0&0&0\\0&1&0&0\\0&0&-1&0\\0&0&0&1\end{bmatrix}\;\; z\mapsto\begin{bmatrix}0&0&0&1\\0&0&1&0\\0&1&0&0\\1&0&0&0\end{bmatrix}.
    \]
    
    \item[]<3->\hrule
    \item[]<3-> And on $\mathbb{C}_{-1}[v_1,v_2]$ via the rep
    \[
    x\mapsto \begin{bmatrix}0&1\\1&0\end{bmatrix}\;\; y\mapsto\begin{bmatrix}0&-1\\-1&0\end{bmatrix}\;\; z\mapsto\begin{bmatrix}1&0\\0&-1\end{bmatrix}.
    \]
\end{itemize}
\end{frame}

\begin{frame}{Quantized Universal Enveloping Algebra}

%From: Walton, Witherspoon "Poincare-Birkhoff-Witt deformations.." 2014

Described by P. Kulish and N. Reshetikhin in``Quantum linear problem..." (1983), leading Vladimir Drinfeld to quantum groups

$\mathcal{U}_q(\mathfrak{sl}
_2)=$
\[
\Big\langle E,F,K,K^{-1}\;\vert\;EF-FE=(q-q^{-1})^{-1}\left(K-K^{-1}\right), KEK^{-1}=q^2 E, \]\[\hspace{4ex}KFK^{-1}=q^{-2}F, KK^{-1}=K^{-1}K=1\Big\rangle
\]
with operations:
\vspace{0.5ex}
\begin{itemize}
\setlength{\itemsep}{1.5ex}
    \item[]<2-> $\triangle(E)=E\otimes 1+K\otimes E,\; \triangle(F)=F\otimes K^{-1}+ 1\otimes F$, \item[]<2-> $\triangle(K)=K\otimes K,\; \triangle(K^{-1})=K^{-1}\otimes K^{-1}$,
    \item[]<3-> $\varepsilon(E)=\varepsilon(F)=0,\; \varepsilon(K)=\varepsilon(K^{-1})=1$,
    \item[]<4-> $S(E)=-K^{-1}E,\; S(F)=-F K, \;S(K)=K^{-1},\; S(K^{-1})=K$.
\end{itemize}
\onslide<5->{Note: You can recover $\mathcal{U}(\mathfrak{sl_2})$ by limiting $q\to 1$.}

%See: Brown, Goodearl, "Lectures on Algebraic Quantum Groups" 2002

\end{frame}

\begin{frame}{Actions of $\mathcal{U}_q(\mathfrak{sl}_2)$}

$\mathcal{U}_q(\mathfrak{sl}_2)$ acts on $\mathbb{C}_q[v_1,v_2]$ by the representation
\begin{columns}[T]
\begin{column}{0.5\textwidth}
\begin{align*}
\onslide<2->{
E & \mapsto \begin{bmatrix} 0&1\\0&0\end{bmatrix}\\} 
\onslide<3->{
K & \mapsto\begin{bmatrix}q&0\\0&q^{-1}\end{bmatrix}} 
\end{align*}
\end{column}
\begin{column}{0.5\textwidth}
\begin{align*}
\onslide<2->{F & \mapsto \begin{bmatrix} 0&0\\1&0\end{bmatrix}\\}
\onslide<3->{K^{-1} & \mapsto \begin{bmatrix}q^{-1}&0\\0&q\end{bmatrix}}
\end{align*}
\end{column}
\end{columns}

\end{frame}

\begin{frame}[fragile]{Hopf Algebra}

A \textbf{Hopf algebra} is a bialgebra $H$ over a field with an antipode \\\;\;\;\;\;\;\;\;$\color{darkblue}S:H\to H$ \;\;\;where the bialgebra operations are 
\vspace{-2ex}
\begin{multicols}{2}
\begin{itemize}
\item[]<2->$\color{lightred}\bigtriangledown:H\otimes H\to H$, 
\item[]<3->$\color{lightgreen}\;1_H:\mathbb{C}\to H$, 
\item[]<4->$\color{darkred}\triangle:H\to H\otimes H$, 
\item[]<5->$\color{darkgreen}\;\varepsilon:H\to\mathbb{C}$
\end{itemize}
\end{multicols}
\vspace{-2ex}
\pause so that the following commute:

\begin{minipage}[t]{0.3\textwidth}
\begin{itemize}
    \item[]<2->\text{Associativity:}
\begin{tikzcd}[color=lightred,cramped]
H\otimes H\otimes H \arrow[d, "id\otimes \bigtriangledown"'] \arrow[r, "\bigtriangledown\otimes id"] & H\otimes H \arrow[d, "\bigtriangledown"] \\
H\otimes H \arrow[r, "\bigtriangledown"']                                                         & H                                    
\end{tikzcd}
\rule{10cm}{0.4pt}
\item[]<3->\text{Unit:}
% https://tikzcd.yichuanshen.de/#N4Igdg9gJgpgziAXAbVABwnAlgFyxMJZARgBoAGAXVJADcBDAGwFcYkQBBAHS4jwFt4AAg4gAvqXSZc+QinKli1Ok1bseAaw08+WQXBHjJIDNjwEiZAEzKGLNok5GpZ2USuLbqh5x0DhmhriyjBQAObwRKAAZgBOEPxIZCA4EEgKKvbqXDA49H56wlhQziBxCUgAzDSp6TR2ao7FBfo8ufQgNIz0AEYwjAAK0uZyILFYYQAWOKXliYjJtYgemY0gALydIN19g8NujuNTMxIx8fPVKWnL9d7sm6dl53VXSCsNPjw44-RgYYyhCAAd0IYkoYiAA
\begin{tikzcd}[cramped,sep=small,color=lightgreen]
& H\otimes H \arrow[dd, "\bigtriangledown"] &                                                         \\
\mathbb{C}\otimes H \arrow[ru, "1_H\otimes id"] \arrow[rd, "="'] &                    & H\otimes \mathbb{C} \arrow[lu, "id\otimes 1_H"'] \arrow[ld, "="] 
                \\
& H &                                                    
\end{tikzcd}
\end{itemize}
\end{minipage}~\hspace{2.2cm}\hfill\vline\hfill\begin{minipage}[t]{0.48\linewidth}
    \begin{itemize}
        \item[] <4->\text{Coassociativity:} % https://tikzcd.yichuanshen.de/#N4Igdg9gJgpgziAXAbVABwnAlgFyxMJZABgBpiBdUkANwEMAbAVxiRAAkQBfU9TXfIRQBGclVqMWbdgB0ZEPAFt4AAk48+2PASJlh4+s1aIOchVmVw13XiAxbBRUfuqGpJ2fKWrP5y9a5xGCgAc3giUAAzACcIRSQyEBwIJFEJIzY5HGisOjAQhlYNEBi4hOpkpAAmV0ljECycvIKi21L4xBqklMQAZlqMk0bc-MKQajgACyxInATi9tSKnv709waZbJGWs28rLChuCi4gA
\begin{tikzcd}[color=darkred,cramped]%[sep=small]
H \arrow[r, "\triangle"] \arrow[d, "\triangle"'] & H\otimes H \arrow[d, "\triangle\otimes id"] \\
H\otimes H \arrow[r, "id\otimes\triangle"']               & H\otimes H\otimes H        
\end{tikzcd}
        \item[]<5->\text{Counit:}
% https://tikzcd.yichuanshen.de/#N4Igdg9gJgpgziAXAbVABwnAlgFyxMJZATikZ package - Overleaf, Online LaTeX EditorBgBoBGAXVJADcBDAGwFcYkQAdDga264jwBbeAAIAEiAC+pdJlz5CKcqWLU6TVuwnTZ2PASIAmCmoYs2iEGP5D4XXlJkgMehUWWHTGi1ZtZhcOJSajBQAObwRKAAZgBOEIJIAMw0OBBIZOrm7FwMsTBo2IwEfgEiWFCOMfGJiMogaRk0ZpqWALwgNIz0AEYwjAAKcvqKILFYYQAWOFUgcQlI9Y2IxlmtIB06czXJqekrzd7sFaV2HHkFRQZdvf1DrgaW41MzW-O1S-spaz5cOOP0MBhRhsSSUSRAA
\begin{tikzcd}[cramped,sep=small,color=darkgreen]
             & H \arrow[ld, "="'] \arrow[rd, "="] \arrow[dd, "\triangle"]                         &             \\
\mathbb{C}\otimes H &                                                                                    & H\otimes\mathbb{C} \\
             & H\otimes H \arrow[lu, "\varepsilon\otimes id"] \arrow[ru, "id\otimes\varepsilon"']&            
\end{tikzcd}
    \end{itemize}
\end{minipage}

\end{frame}

\begin{frame}[fragile]{Hopf Algebra Diagrams}

\begin{itemize}
\item[]<1->\text{Product and Coproduct compatibility:}
% https://tikzcd.yichuanshen.de/#N4Igdg9gJgpgziAXAbVABwnAlgFyxMJZABgBpiBdUkANwEMAbAVxiRAAkAdTiPAW3gACdiAC+pdJlz5CKAIzkqtRizYjxk7HgJEATIur1mrRB268sAuMLESQGLTKJk5So6tNce-IV4tXhcx9rdTsHaR0UfVdDFRMzb0tfIKSQlID1JRgoAHN4IlAAMwAnCD4kMhAcCCQAZljjNm4cYqw6MByGGHT4Ztb2ztYNEBKyuupqpAAWBo8QLCgeuGa6JiXBBdsi0vLESsnEBWVG0z62jq6oCAB3QmHR3aOD-WO5s4GurZGd6YmaxBe7ni7wu2RuYCWIMGV1uYgooiAA
\begin{tikzcd}[color=lightred]
H\otimes H \arrow[d, "\triangle\otimes\triangle"'] \arrow[r, "\bigtriangledown"] & H \arrow[r, "\triangle"] & H\otimes H                                                                  \\
H\otimes H\otimes H\otimes H \arrow[rr, "id\otimes\tau\otimes id"']           &                          & H\otimes H\otimes H\otimes H \arrow[u, "\bigtriangledown\otimes\bigtriangledown"']
\end{tikzcd}

\item[]<2->\rule{8cm}{0.4pt}
\text{Unit and Counit compatibility:}
% https://tikzcd.yichuanshen.de/#N4Igdg9gJgpgziAXAbVABwnAlgFyxMJZABgBpiBdUkANwEMAbAVxiRAB12BbOnACwBGA4AGEAviDGl0mXPkIoAjOSq1GLNgAlJ0kBmx4CRZYtX1mrRCE2cIeLvAAE2qTIPyiAJhXVzGqy66+nJGKN6mvuqW1rb2ToFuIQrIAMw+ahZsnDz8QqISYqowUADm8ESgAGYAThBcSGQgOBBIyhn+HOwwOHQ6VbX1iI3NSN4gDHQCMAwACrKGCiDVWCV8OCCRmVac3XSxWA5wOz19IDV1rdQjiGN+0Zw4y3RgJQys1HB8WJXriG0TU1m8w8VjeP1O50GABYri1EGl2vd2I8sM9XqxXGcBkgEdcAKybDqcejVGBobAMIyYyFIGFNOEExFZdgkskUozUAHTObuUJLFZrSQUMRAA
%\begin{center}
\begin{tikzcd}[color=darkgreen]
\mathbb{C} \arrow[r, "1_H"] \arrow[rd, "1_H\otimes 1_H"'] & H \arrow[d, "\triangle"] & \phantom{a} & H \arrow[r, "\varepsilon"]                                   & \mathbb{C} \\
                                                            & H\otimes H                            & \phantom{a} & H\otimes H \arrow[u, "\bigtriangledown"] \arrow[ru, "\varepsilon\otimes\varepsilon"'] &           
\end{tikzcd}
%\end{center}
\item[]<3->\rule{8cm}{0.4pt}\\
\text{Antipode:}
% https://tikzcd.yichuanshen.de/#N4Igdg9gJgpgziAXAbVABwnAlgFyxMJZABgBoBGAXVJADcBDAGwFcYkQAJEAX1PU1z5CKchWp0mrdgB1pAazk8+IDNjwEiAJjE0GLNok5L+aoUTKbxeqYY6yIeALbwABF14nBGlNsu7JBpz2Tq7uyqpewiSkxFYB7HbSDljOcG7GKgLqUdqx-voJwSmhPOIwUADm8ESgAGYAThCOSGQgOBBIAKz5NiCyOPVY9GAVjGweIA1NXTTtSABsPYFYUEWpLgDKGVPNiIttHYjaEgWG-YPDo+UQAO6EEzsts4cAzEsy0gNDI2MgNIz0ABGMEYAAUsmZDIMKgALHDbRq7N4HJAAFnehg2a1cKwR00Q6JRRwxfU+Fx+1zueN2rTmiFEJ16sgY9RgaGwjA0D0RSAZdOO1kCshgOHopW4QA
\begin{tikzcd}[color=darkblue]
H\otimes H \arrow[rr, "id\otimes S"]                                      &                       & H\otimes H \arrow[d, "\bigtriangledown"] \\
H \arrow[u, "\triangle"] \arrow[d, "\triangle"'] \arrow[r, "\varepsilon"] & \mathbb{C} \arrow[r, "1_H"] & H                                     \\
H\otimes H \arrow[rr, "S\otimes id"]                                      &                       & H\otimes H \arrow[u, "\bigtriangledown"']
\end{tikzcd}
\end{itemize}
\end{frame}

\begin{frame}[fragile]{Hopf Algebra Actions}

Let $H$ be a Hopf alg and $A$ an alg with a map $\alpha:H\otimes A\to A$. Then we say $H$ \textbf{acts} on $A$ by $\alpha$ if the following diagrams commute:

\begin{itemize}
    \item[]<2->\begin{tikzcd}[color=lightred]
H\otimes H\otimes A \arrow[r, "\bigtriangledown\otimes id"] \arrow[d, "id\otimes \alpha"] & H\otimes A \arrow[d, "\alpha"] \\
H\otimes A \arrow[r, "\alpha"]                                                         & A                             
\end{tikzcd}
% https://tikzcd.yichuanshen.de/#N4Igdg9gJgpgziAXAbAABwnAlgFyxMJZABgBpiBdUkANwEMAbAAxiRAB12BrLziPALbwABADUQAX1LpMufIRQBGclAqMWbABJ9BI8AJnY8BIssWr6zAohD7AMKAHN4RUADMAThAFIyIHBBIympWbJwwOHQ6WEJwwlhQktIgnt6+1AFIAEzUlho2ALwg1Ax0AEYwDAAKssYKIB5YjgAWOEnuXj6IwZmIOSH5HOyMaM10khQSQA  
        \begin{tikzcd}[color=darkgreen]
\mathbb{C}\otimes A \arrow[r, "1_H\otimes id"] \arrow[rd, "="'] & H\otimes A \arrow[d, "\alpha"] \\
                                                          & A                                                                       
\end{tikzcd}
\vspace{1.5ex}
\end{itemize}
\pause \pause $A$ is called a \textbf{module algebra} if the following also commute:
\begin{itemize}
\item[]<4-> \begin{tikzcd}[color=darkblue]
H\otimes A\otimes A \arrow[r, "\bigtriangledown"] \arrow[d, "\triangle\otimes id\otimes id"'] & H\otimes A \arrow[r, "\alpha"] & A & A\otimes A \arrow[l, "\bigtriangledown"']                         \\
H\otimes H\otimes A\otimes A \arrow[rrr, "id\otimes\tau\otimes id"]                        &                                &   & H\otimes A\otimes H\otimes A \arrow[u, "\alpha\otimes\alpha"']
\end{tikzcd}.\
% https://tikzcd.yichuanshen.de/#N4Igdg9gJgpgziAXAbVABwnAlgFyxMJZABgBpiBdUkANwEMAbAVxiRAAkQBfU9TXfIRQBGclVqMWbADrSA1nO68QGbHgJEATGOr1mrRCACCSvmsFEyw8XqmH2siHgC28AAQmu4mFADm8IlAAMwAnCGckMhAcCCRRCX0ZaXoQmDRsBg0eYLCIxHiYpG0EuxBZGBw6UxBQ8MjqQsQAZl1JAzLpCqrsmtykFujYxGLbdtlGNAALboouIA
    \item[]<5->
     \begin{center}\begin{tikzcd}[color=darkgreen]
H \arrow[r, "\varepsilon"] \arrow[d, "-\otimes 1_A"'] & \mathbb{C} \arrow[r, "1_A"] & A \\
H\otimes A \arrow[rru, "\alpha"']             &                       &  
\end{tikzcd}\end{center}
\end{itemize}
\end{frame}

\begin{frame}{Hopf Alebra Actions}
In words, $H$ acts on $A$ iff you can multiply in $H$ and then act on $A$ or act on $A$ consecutively, $\forall h,h'\in H,\forall a\in A$
\[
(hh')(a)=h(h'(a)),\;\;\; 1_H(a)=a. 
\]
And $A$ is an $H$-module alg iff $H$ acts on $A$ and $\forall h\in H,\forall a,a'\in A$
\[
h(aa')=\sum h_i(a)\cdot h_j(a),\;\;\; h(1_A)=\varepsilon(h)1_A    
\]
where $\triangle(h)=\sum h_i\otimes h_j$.

\end{frame}

\begin{frame}{Semidirect Product}
    Let $G$ and $G'$ be groups where $G'$ acts on $G$ by automorphisms, giving the semidirect product group, $G\rtimes G'$.\\
    The action can be extended to the group algebras:
    \[
    \mathbb{C}(G\rtimes G')=\mathbb{C}G\#\mathbb{C}G'    
    \]
    with product $g'g=g'(g)g'$.\\
    \begin{block}{Smash Product Algebra}
        If $H$ is a Hopf algebra and $A$ an $H$-module algebra, then $A\#H$ is defined as $A\otimes H$ as a vector space and with product
        \[
        (a'\otimes h)(a\otimes h')=\sum_{i} a'h_{i_1}(a)\otimes h_{i_2}h' 
        \]
        where $a\in A$, $h\in H$ and $\triangle(h)=\sum_{i} h_{i_1}\otimes h_{i_2}$.
    \end{block}
\end{frame}

\begin{frame}{Smash Product Algebra}
    \begin{block}{"Group-like" and "Lie-like"}
        For Hopf alg $H$, define \[G(H)=\{h\in H\;\vert\; \triangle(h)=h\otimes h\}=\text{grouplike elements}\] 
        which forms a group, and define 
        \[P(H)=\{h\in H\;\vert\; \triangle(h)=h\otimes 1+1\otimes h\}=\text{primitive elements}\]
        which forms a Lie alg.
    \end{block}
    \begin{block}{Cartier-Kostant-Milnor-Moore Theorem}
        Let $H$ be a cocommutative Hopf algebra over $\mathbb{C}$, then as Hopf algebras,
        \[
            H\cong \mathcal{U}\left(P(H)\right)\# \mathbb{C}G(H)
        \]
    \end{block}
    Cor: Any cocomm, finite-dimn'l Hopf alg over $\mathbb{C}$ is iso to a grp alg.
\end{frame}

\begin{frame}{Research Directions}
    \begin{itemize}
    \setlength{\itemsep}{2ex}
        \item<1-> Which actions on algebras by Hopf algebras factor through a group action?
                    If the action does not factor through a group action, it is said to have "quantum symmetry".
        \item<2-> Which Hopf Algebras act on AS-regular algebras?
        \item<3-> When are the invariant subrings from Hopf actions AS-Gorenstein?
        \item<4-> If $H$ is semisimple and finite-dimensional, and $A$ is semiprime, is $A\#H$ semiprime?
        \item<5-> If $B$ is a Koszul algebra, are there nontrivial PBW deformations of $B\# \mathcal{U}_q(\mathfrak{sl_2})$?
    \end{itemize}
\end{frame}

\end{document}
