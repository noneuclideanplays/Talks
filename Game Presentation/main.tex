\documentclass{beamer}
\usepackage{amsthm,amsmath,amsfonts, amssymb, ulem}

\usepackage{graphicx}
\usepackage{ulem}
\usepackage{untgreen}
\UNTGpalette
\usetheme{Warsaw}

\newtheorem{conjecture}{Conjecture}


\title{A Game Considered on $D_n$}
\author{Logan Crone, Brandon Mather, Erin Raign, Julie Thompson}
\date{10.1.21}


\begin{document}

\begin{frame}

\titlepage

\end{frame}
    
\begin{frame}{Background}
   \begin{itemize}
       \item $D_n$ is a group made up of rotations, $\rho^n$, and reflections, $\mu \rho^n$. The order of $D_n$ is $2n$. \bigskip
       \item The conjugacy class of the element $x\in D_n$ is $cl(x) = \{gxg^{-1} : g\in D_n\}$.\\ \bigskip
       
       \item The conjugacy classes partition the elements of $D_n$, since for all $a,b\in D_n, cl(a)=cl(b)$ iff $a$ and $b$ are conjugate; they are disjoint otherwise.
      
   \end{itemize} 
   
\end{frame}

\begin{frame}{How do we play the game?}
\begin{itemize}
    \item We will play the game on the dihedral group, $D_n$.
    \medskip
     \item We will play from the conjugacy classes of $D_n$, excluding the identity, $e$.
     \medskip
    \item After an element from a conjugacy class has been played, no element from that conjugacy class can be played for the rest of the game.
    \medskip
    \item After each element is played, we take the product of all moves that have been played.
\end{itemize}
\end{frame}

\begin{frame}{How do we play the game?}
    \begin{itemize}
        \item The goal is to make a move which changes the product to $e$, the identity element; this player wins the game.
        \medskip
        \item If all conjugacy classes have been played before the product equals $e$, the game is a draw.
        \medskip
        \item A winning strategy for a player is an algorithm that a player follows to always win the game.
        \medskip
        \item A good move for a player is a move where they will not lose after the next player's turn.
    \end{itemize}
\end{frame}


\begin{frame}{Example}
   Let $G = D_4$.\\ \bigskip
   $D_4 = \{e, \rho, \rho^2, \rho^3, \mu, \mu \rho, \mu \rho^2, \mu \rho^3\}$.\\ \bigskip
   The conjugacy classes of $D_4$ are: $\{e\}, \{\rho, \rho^3\}, \{\rho^2\}, \{\mu, \mu\rho^2 \}, \{\mu\rho, \mu\rho^3 \}$.
\end{frame}

\begin{frame}{Example game played in $D_4$}
\begin{center}
\only<1>{$\{\rho, \rho^3\}, \{\rho^2\}, \{\mu, \mu\rho^2 \}, \{\mu\rho, \mu\rho^3 \}$} 
\begin{sloppypar}
\only<2>{{\sout{$\{\rho, \rho^3\}$}, $\{\rho^2\}, \{\mu, \mu\rho^2 \}, \{\mu\rho, \mu\rho^3 \}$}}
\end{sloppypar}
\begin{sloppypar}
\only<3>{\sout{$\{\rho, \rho^3\}$}, \sout{$\{\rho^2\}$}, $\{\mu, \mu\rho^2 \}, \{\mu\rho, \mu\rho^3 \}$}
\end{sloppypar}
\begin{sloppypar}
\only<4>{\sout{$\{\rho, \rho^3\}$}, \sout{$\{\rho^2\}$}, \sout{$\{\mu, \mu\rho^2 \}$}, $\{\mu\rho, \mu\rho^3 \}$}
\end{sloppypar}
\begin{sloppypar}
\only<5>{\sout{$\{\rho, \rho^3\}$}, \sout{$\{\rho^2\}$}, \sout{$\{\mu, \mu\rho^2 \}$}, \sout{$\{\mu\rho, \mu\rho^3 \}$}}
\end{sloppypar}

   
   \begin{tabular}{c|ccccc}
    \hline
        Player 1 & \onslide<2->{$\rho$} & & \onslide<4->{$\mu$} \\
        Player 2 & & \onslide<3->{$\rho^2$} & &  \onslide<5->{$\mu\rho$}\\
    \hline 
        Current Product & \onslide<2->{$\rho$} & \onslide<3->{$\rho^3$} & \onslide<4->{$\mu\rho$} & \onslide<5->{$e$}
    \end{tabular}
    \end{center}
\end{frame}

\begin{frame}{}
    \begin{block}{Definition}
    Let $G$ be a finite group. We play the following game:
    \medskip
    
    \begin{tabular}{c|ccc}
    \hline
        Player 1 & $g_0 \in G$ & & $g_2 \in G \backslash (cl\{g_0\} \cup cl\{g_1\})$ \\
        Player 2 & & $g_1 \in G \backslash cl\{g_0\}$ 
    \end{tabular}\\
    \medskip
    where $g_n \in G\backslash \bigcup_{i<n} cl\{g_i\}$ and $g_n \not = e$. Define $p_n = g_0 g_1 \cdot \cdot \cdot g_n$.\\
    If both players run out of moves, it is a draw. Otherwise, the first player to make $p_n = e$ wins.
    \end{block}
\end{frame}

\begin{frame}{Motivation}
    \begin{itemize}
        \item There are few games played on algebraic groups.
        \medskip 
        \item We can learn about the object the game is played on.
        \medskip
        \item This can be played on other algebraic groups.
        \medskip
        \item Why play from conjugacy classes?
        \medskip
        \item Why play on $D_n$?
    \end{itemize}
\end{frame}

\begin{frame}{Results}

\begin{Theorem}[C., M., R., T.]
    \begin{itemize}
        \item If $n\equiv 1 \mod 2$, then both players can force a draw.
        \item If $n\equiv 2 \mod 4$, then Player 1 has a winning strategy.
        \end{itemize}
\end{Theorem}
\begin{conjecture}
        If $n\equiv 0 \mod 4$, we conjecture the winning player alternates depending on $n\mod 8$.
\end{conjecture}
    
\end{frame}
\begin{frame}{Strategy for $n \equiv 1 \mod 2$}
When $n$ is odd, both players can always force a draw.\\
\medskip
Note there is exactly one conjugacy class containing reflection elements when $n$ is odd.\\
\medskip
Both players can force a draw by playing a reflection element.\\
\bigskip
Let G=$D_5$\\
Conjugacy classes: $\{e\}, \{\rho,\rho^4\},\{\rho^2,\rho^3\},\{\mu,\mu\rho, \mu\rho^2, \mu\rho^3, \mu\rho^4\}$.\\
    \begin{tabular}{c|ccccc}
    \hline
        Player 1 & {$\rho$} & & {$\rho^2$} \\
        Player 2 & & {$\mu$} & \\
    \hline 
        Current Product & {$\rho$} & {$\mu\rho^4$} & {$\mu\rho$}
    \end{tabular}\\
\end{frame}

\begin{frame}{$n\equiv 2 \mod 4$}
When $n\equiv 2 \mod 4$, Player 1 has a winning strategy. \\
\medskip
Player 1's winning strategy  is to play the center, $\rho^{\frac{n}{2}}$.\\
\medskip
Then Player 2 has no good move. 
\bigskip

Let $G=D_6$. \\
\medskip
Conjugacy classes: $\{e\},\{\rho^3\}, \{\rho, \rho^5\}, \{\rho^2\, \rho^4\}, \{\mu, \mu\rho^2, \mu\rho^4\}, \{\mu\rho, \mu\rho^3, \mu\rho^5\}$. \\
\medskip


 \begin{tabular}{c|ccc}
    \hline
        Player 1 & $\rho^3$ & & $\mu\rho^4$ \\
        Player 2 & & $\mu\rho$ \\
    \hline
    Current Product & & $\mu\rho^4$ & $e$ 
    \end{tabular}\\
\end{frame}

\begin{frame}{$n\equiv 2 \mod 4$ Example of a Bad Move}
Let $G=D_6$. \\
\medskip
Conjugacy classes: \\
\medskip
$\{e\},\{\rho^3\}, \{\rho, \rho^5\}, \{\rho^2\, \rho^4\}, \{\mu, \mu\rho^2, \mu\rho^4\}, \{\mu\rho, \mu\rho^3, \mu\rho^5\}$ \\
\medskip
 \begin{tabular}{c|cccc}
    \hline
        Player 1 & $\rho^2$ & & $\rho$ \\
        Player 2 & & $\mu\rho$ & & $\mu$\\
    \hline
    Current Product & & $\mu\rho^5$ & $\mu$ & $e$
    \end{tabular}\\
\end{frame}

\begin{frame}{$n\equiv 0 \mod 4$}

\begin{conjecture}
If $n\equiv0 \mod 4$, we split further into two cases: \\
\begin{itemize}
    \item If $n\equiv 0 \mod 8$, then Player 1 has a winning strategy.\\
    \item If $n\equiv 4 \mod 8$, then Player 2 has a winning strategy. 
\end{itemize}
\bigskip

Furthermore, if there is an even number of even powered rotation classes, Player 1 has a winning strategy. \\
\medskip
If there is an odd number of even powered rotation classes, Player 2 has a winning strategy. 
\end{conjecture}
\end{frame}

\begin{frame}{Game played on $D_{8}$}
    The conjugacy classes are:\\ $\{e\}$, $\{\rho,\rho^{7}\}$, $\{\rho^2,\rho^{6}\}$,$\{\rho^3,\rho^5\}$,$\{\rho^4\}$, $\{\mu, \mu\rho^2, \mu\rho^4, \mu\rho^6\}$ and $\{\mu\rho, \mu\rho^3, \mu\rho^5, \mu\rho^7\}$\\
    \bigskip
    Note that the even powered rotation classes for $D_8$ are $\{\rho^{2}, \rho^{6}\}$ and $\{\rho^4\}$, so there are two of them (an even number). We see Player 1 has a winning strategy:\\
    \bigskip
    \begin{tabular}{c|cccccc}
    \hline
        Player 1 & {$\rho^2$} & & {$\mu\rho^2$} & & {$\mu\rho^5$}\\
        Player 2 & & {$\rho^4$} & & {$\rho$} \\
    \hline 
        Current Product & {$\rho^2$} & {$\rho^6$} & {$\mu\rho^4$} & {$\mu\rho^5$} & {$e$}
    \end{tabular}\\
\end{frame}

\begin{frame}{Game played on $D_{12}$}
    The conjugacy classes are:\\ $\{e\}$, $\{\rho,\rho^{11}\}$, $\{\rho^2,\rho^{10}\}$,$\{\rho^3,\rho^9\}$,$\{\rho^4,\rho^8\}$,$\{\rho^5,\rho^7\}$, $\{\rho^6\}$,$\{\mu, \mu\rho^2, \mu\rho^4, \mu\rho^6, \mu\rho^8, \mu\rho^{10}\}$ and $\{\mu\rho, \mu\rho^3, \mu\rho^5, \mu\rho^7, \mu\rho^9, \mu\rho^{11}\}$\\
    \bigskip
    Note that the even powered rotation classes for $D_{12}$ are $\{\rho^2,\rho^{10}\}$, $\{\rho^4,\rho^8\}$, and $\{\rho^6\}$, so there are three of them (an odd number). We see Player 2 has a winning strategy: \\
    \bigskip
    \begin{tabular}{c|cccccc}
    \hline
        Player 1 & {$\rho^4$} & & {$\rho^6$} & & {$\rho$}\\
        Player 2 & & {$\rho^{10}$} & & {$\mu$} & & {$\mu\rho^5$}\\
    \hline 
        Current Product & {$\rho^4$} & {$\rho^2$} & {$\rho^8$} & {$\mu\rho^4$} & {$\mu\rho^5$} & {$e$}
    \end{tabular}\\
\end{frame}

\begin{frame}{Motivation for the Game on $\mathbb{Z}_n$}
\begin{itemize}
\item We noticed that when a reflection class was played, it played a vital role in the outcome of the game.
\medskip
\item This led to players playing strictly from the rotation classes until all even rotation classes had been exhausted. 
\medskip
\item Can both players force a draw playing only from the rotation classes?
\medskip
\item This led to an auxiliary game on $\mathbb{Z}_n$.
\end{itemize}
\end{frame}

\begin{frame}{The game on $\mathbb{Z}_n$}
Let $G=(\mathbb{Z}_n, +)$.The identity element in $(\mathbb{Z}_n, +)$ is 0.
\medskip
\\Notice since $\mathbb{Z}_n$ is abelian, the conjugacy class of $z$ for any $z\in\mathbb{Z}_n$ is just itself,$\{z\}$.
\medskip
\\In order to replicate the game on $D_n$, we let the sets the game is played on be $\{\pm{a}_i|{a}_i\in\mathbb{Z}_n\}\cup\{0\}$ for each $i\in\{1,\dots,n\}$.


\end{frame}

\begin{frame}{Game played on $\mathbb{Z}_n$}
    $\mathbb{Z}_8=\{0,1,2,3,4,5,6,7\}$ \\ 
    Move classes: $\{0\},\text{ }\{1,7\}, \text{ } \{2,6\},\text{ }  \{3,5\}, \text{ } \{4\}$\\
    Illegal moves: $\{0\}$\onslide<1->{$, \{1,7\}$}\onslide<2->{, $\{2,6\}$}\onslide<3->{, $\{3,5\}$}\onslide<4->{, $\{4\}$}
    \bigskip
    
    \begin{tabular}{c|cccccc}
    \hline
        Player 1 & \onslide<1->{1} & & \onslide<3->{3} \\
        Player 2 & & \onslide<2->{6} & &  \onslide<4->{4}\\
    \hline 
        Current Sum && \onslide<2->{7} & \onslide<3->{2} & \onslide<4->{6} 
    \end{tabular}\\
    \bigskip
    \onslide<4->{The game ends in a draw.}
\end{frame}
    

\begin{frame}{Going Back to the Game on $n \equiv 0 \mod 4$}
    \begin{itemize}
        \item We hope to show both players can force a draw in the $\mathbb{Z}_n$ game.
        \medskip
        \item Then the player's winning strategy, when $n\equiv 0 \mod 4$, will depend on when a reflection class is played.
        \medskip
        \item We will finish our characterization of $D_{n}$ under this game.
    \end{itemize}
\end{frame}

\begin{frame}{Future Work}
  \begin{itemize}
      \item We plan to consider other groups on this game.
      \medskip
      \item The game may illuminate some group theoretic properties.
  \end{itemize}  
\end{frame}

\end{document}