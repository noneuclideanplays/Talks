\documentclass[12pt,a4paper]{article}
\usepackage[english]{babel}
\usepackage[utf8x]{inputenc}
\usepackage[T1]{fontenc}
\usepackage{listings}
\usepackage{amsfonts}
\usepackage{amssymb}
\usepackage{amsmath,amsthm}
\usepackage{mathtools}
\usepackage{graphicx}
\usepackage{geometry}
\usepackage{layout}
\usepackage{tikz}
\usetikzlibrary{positioning}
\newcommand{\comp}{\overline{\phantom{A}}}
\setlength{\tabcolsep}{12pt}

\newtheorem{theorem}{Theorem}[section]
\newtheorem{lemma}[theorem]{Lemma}
\newtheorem{claim}[theorem]{Claim}
\newtheorem{proposition}[theorem]{Proposition}
\newtheorem{corollary}[theorem]{Corollary}
\newtheorem{fact}[theorem]{Fact}
\newtheorem{example}[theorem]{Example}
\newtheorem{notation}[theorem]{Notation}
\newtheorem{observation}[theorem]{Observation}
\newtheorem{conjecture}[theorem]{Conjecture}
\newtheorem{remark}[theorem]{Remark}
\newtheorem{definition}[theorem]{Definition}
\newcommand\NN{\mathbb{N}} 
\newcommand\RR{\mathbb{R}}
\newcommand\CR{\mathcal{R}}
\newcommand\ZZ{\mathbb{Z}}
\newcommand\AL{\mathcal{A}}
\newcommand\MM{\mathcal{M}}
\newcommand\BB{\mathcal{B}}
\newcommand\BBB{\mathbf{B}}
\newcommand\CC{\mathcal{C}}
\newcommand\CCC{\mathfrak{C}}
\newcommand\BC{\mathbb{C}}
\newcommand\DD{\mathcal{D}}
\newcommand\FF{\mathcal{F}}
\newcommand\GG{\mathcal{G}}
\newcommand\HH{\mathcal{H}}
\newcommand\JJ{\mathcal{J}}
\newcommand\PP{\mathcal{P}}
\newcommand\QQ{\mathbb{Q}}
\newcommand\CS{\mathcal{S}}
\newcommand\TT{\mathcal{T}}
\newcommand\BT{\mathbb{T}}
\newcommand\OO{\mathcal{O}}
\newcommand\UU{\mathcal{U}}
\newcommand\VV{\mathcal{V}}
\newcommand\PPT{\mathcal{P}^{*}}
\newcommand\CST{\Sigma^{*}}
\newcommand\RX{\mathsf{X}}
\newcommand\LL{\mathcal{L}}

%\voffset=-1truein		% LaTeX has too much space at page top
\addtolength{\textheight}{0.3truein}
\addtolength{\textheight}{\topmargin}
\addtolength{\topmargin}{-\topmargin}
\textwidth  6.0in		% LaTeX article default 360pt=4.98''
%\oddsidemargin 0pt	% \oddsidemargin  .35in   % default is 21.0 pt
%\evensidemargin 0pt	% \evensidemargin .35in   % default is 59.0 pt
%\marginparwidth=0pt
% decrease margins on sides and top/bottom
\geometry{
bottom=20mm,
margin=20mm
}
\title{Determining Decidability Algebraically}
\date{}
\begin{document}%\layout
	%\MakeScribeTop
	

	
\maketitle

\begin{abstract}
In this talk we will contextualize common algebraic notions in Universal Algebra and provide examples of translations into Universal Algebra.
Then we will discuss the logical notions of Universal Algebra and how to use these to perform Model Theoretic proofs.
Finally we will show how to use Universal Algebra to show the undecidiability of certain theories, in particular the theory of groups.
\end{abstract}
\section{Algebras}

\begin{definition}
Let $A\neq\emptyset$ be a set and $n\in\mathbb{Z}$. An \textbf{$n$-ary operation/function} on $A$ is a function $f:A^n\to A$
\end{definition}

\begin{definition}
A \textbf{language/type} of algebras is a set $\mathcal{F}$ of function symbols such that each function has an arity
\end{definition}

\begin{definition}
Let $\mathcal{F}$ be a language of algebras, then an \textbf{algebra} $\mathbb{A}$ \textbf{of type } $\mathcal{F}$ is an ordered pair $\left<A,F\right>$ where $A$ is a non-empty set and $F$ is a family of finitary operations on $A$ indexed by $\mathcal{F}$ such that corresponding to each $n$-ary $f\in\mathcal{F}$ there is an $n$-ary operation $f^\mathbb{A}$ on $A$
\end{definition}

We call $A$ the \textbf{universe} of $\mathbb{A}=\left<A,F\right>$.

\begin{example}
A \textbf{groupoid} or \textbf{magma} is an algebra $\mathbb{G}=\left<G,\cdot\right>$ with only a single binary operation.\\
A \textbf{group} is an algebra $\mathbb{G}=\left<G,\cdot,{}^{-1},1\right>$ such that $x\cdot(y\cdot z)\approx (x\cdot y)\cdot z$, $x\cdot 1\approx 1\cdot x\approx x$ and $x\cdot x^{-1}\approx x^{-1}\cdot x\approx 1$.\\
A \textbf{Boolean algebra} is an algebra $\mathbb{B}=\left<B,\vee,\wedge,',0,1\right>$ such that $\left<B,\vee,\wedge\right>$ is a distributive lattice, $x\wedge 0\approx 0$, $x\vee 1\approx 1$, $x\wedge x'\approx 0$ and $x\vee x'\approx 1$.\\
\end{example}

\begin{definition}
We call $\mathbb{B}$ a \textbf{subalgebra} of $\mathbb{A}$ if $B\subseteq A$ and $f^\mathbb{B}=f^\mathbb{A}\upharpoonright B$.\\
We call $B$ a \textbf{subuniverse} of $\mathbb{A}$ if $B\subseteq A$ and 
\[
f(b_1,\ldots,b_n)\in B  \;\;\;\;\; \text{for all } n\text{-ary } f \text{ and } b_i\in B
\]
\end{definition}

\begin{definition}
Let $\mathbb{A}$ be an algebra and $X\subseteq A$, the \textbf{subuniverse generated by } $X$ is $Sg(X)=\bigcap\{B:X\subseteq B, \text{ B a subuniverse of } \mathbb{A}\}$
\end{definition}
Remark: We can define a lattice $\mathbb{L}_{Sg}$ of subuniverses of $\mathbb{A}$.

\begin{theorem}
If $\mathbb{A}$ is an algebra and $X\subseteq A$, then $|Sg(X)|\le |X|+|\mathcal{F}|+\omega$
\end{theorem}

\section{Algebraic Notions}

\begin{definition}
Let $\mathbb{A}$ be an algebra of type $\mathcal{F}$ and let $\theta$ be an equivalence relation on $A$, $\theta\in Eq(A)$. Then we call $\theta$ a \textbf{congruence} on $\mathbb{A}$ if $\theta$ satisfies the \textbf{compatibility property}: For each $n$-ary function symbol $f\in\mathcal{F}$ and elements $a_i,b_i\in A$, if $a_i\theta b_i$ holds for $1\le i\le n$, then 
\[
f^\mathbb{A}(a_1,\ldots,a_n)\theta f^\mathbb{A}(b_1,\ldots,b_n)
\]
holds.
\end{definition}

In a sense, this means the equivalence relation is compatible with the structure of the group, algebraic operations on equivalent elements yields equivalent elements.

\begin{example}
$\mathbb{Z}=\left<\mathbb{Z},+\right>$ is a group with the congruence relation $\mod n$. If $a\equiv b \mod n$ and $c\equiv d\mod n$ then $a+c\equiv b+d\mod n$.
\end{example}

\begin{definition} The set of all congruence relations on an algebra $\mathbb{A}$ is denoted $\mathbf{Con}\mathbb{A}$. Let $\theta\in\mathbf{Con}\mathbb{A}$, the \textbf{quotient algebra} of $\mathbb{A}$ by $\theta$, written $\mathbb{A}/\theta$, is the algebra whose universe is $A/\theta$ and whose operations satisfy 
\[
f^{\mathbb{A}/\theta}(a_1/\theta,\ldots,a_n/\theta)=f^{\mathbb{A}}(a_1,\ldots,a_n)/\theta
\]
\end{definition}

\begin{definition}
Let $\mathbb{A}$ and $\mathbb{B}$ be two algebras of the same type $\mathcal{F}$, a mapping $\alpha:A\to B$ is called a \textbf{homomorphism} from $\mathbb{A}$ to $\mathbb{B}$ if 
\[
\alpha f^\mathbb{A}(a_1,\ldots,a_n)=f^\mathbb{B}(\alpha a_1,\ldots,\alpha a_n)
\]\\
The \textbf{kernel} of $\alpha$ is a congruence on $\mathbb{A}$ defined by $ker(\alpha)=\{ \left<a,b\right> \in A^2:\alpha(a)=\alpha(b)\}$.\\
\end{definition}

\section{Logical Notions}

\begin{definition}
    We list some important operators on classes of algebras. Let $K$ be a class of algebras and $\mathbf{A}$ an algebra, then:\\
    \begin{tabular}{l}
        $\mathbf{A}\in I(K)\iff\mathbf{A}$ is isomorphic to some member of $K$ \\
        $\mathbf{A}\in S(K)\iff\mathbf{A}$ is a subalgebra of some member of $K$\\
        $\mathbf{A}\in H(K)\iff\mathbf{A}$ is a homomorphic image of some member of $K$\\
        $\mathbf{A}\in P(K)\iff\mathbf{A}$ is a direct product of a nonempty family of algebras in $K$
    \end{tabular}
\end{definition}

\begin{definition}
    A nonempty class $K$ of algebras all of type $\FF$ is called a \textbf{variety} if it is closed under subalgebras, homomorphic images, and direct products.
    We define $V(K)$ to be the smallest variety containing $K$, called the \textbf{variety generated by } $K$.
\end{definition}

\begin{theorem}[Tarski]
    $V=HSP$
\end{theorem}

\begin{definition}
Let $X$ be a set of variables, $\FF$ a type of algebras.
The set $T(X)$ of \textbf{terms of type} $\FF$ \textbf{over X} is the smallest set such that
\begin{align*}
    (i)\; & X\cup \FF_0\subseteq T(X)\\
    (ii) \;& p_1,\ldots,p_n\in T(X), f\in\FF_n\implies f(p_1,\ldots,p_n)\in T(X)
\end{align*}
where $\FF_n$ is the set of $n$-ary function symbols of $\FF$.
\end{definition}

\begin{example}
Given a group $G=\left<G,*,1\right>$ where $G=\{x,y\}$, some terms include $1$, $x$, $x*y$, $1*(x*y)$, some examples of non-terms are $1*$ and $xy$.
\end{example}

\begin{definition}
    An \textbf{identity} of type $\FF$ over $X$ is an expression of the form $p\approx q$ where $p,q\in T(X)$.
    Let $Id(X)$ be the set of identities of type $\FF$ over $X$.
    We say an algebra $\mathbf{A}$ of type $\FF$ \textbf{satisfies} an identity $p(x_1,\ldots,x_n)\approx q(x_1,\ldots,x_n)$, written $\mathbf{A}\models p(x_1,\ldots,x_n)\approx q(x_1,\ldots,x_n)$ or $\mathbf{A}\models p\approx q$ if for any $a_1,\ldots,a_n\in A$ we have $p^\mathbf{A}(a_1,\ldots,a_n)=q^\mathbf{A}(a_1,\ldots,a_n)$. 
    If a every member of a class of algebras $K$ satisfies $p\approx q$ we say $K\models p\approx q$.
    If $\Sigma$ is a set of identities, we say $K\models\Sigma$ if $K\models p\approx q$ for each $p\approx q\in\Sigma$.
    Given $K$ and $X$, let $Id_K(X)=\{p\approx q\in Id(X): K\models p\approx q\}$.
\end{definition}

\begin{definition}
    Let $\Sigma$ be a set of identities of type $\FF$, and define $M(\Sigma)$ to be the class of algebras $\mathbf{A}$ satisfying $\Sigma$.
    A class $K$ of algebras is an \textbf{equational class} if there is a set of identities $\Sigma$ such that $K=M(\Sigma)$, here we say $K$ is \textbf{axiomatized} by $\Sigma$.
\end{definition}

\begin{lemma}
    If $V$ is a variety and $X$ is an infinite set of variables, then $V=M(Id_V(X))$.
\end{lemma}

\begin{theorem}[Birkhoff]
    $K$ is an equational class if and only if $K$ is a variety.
\end{theorem}

\section{Model Theory}

\begin{definition}
    A (first-order) \textbf{language} $\mathcal{L}$ consists of a set $\mathcal{R}$ of \textbf{relation symbols} and a set $\FF$ of \textbf{function symbols}, and associated to each member of $\CR$ and $\FF$ a natural number and nonnegative integers, respectively, called the \textbf{arity} of the symbol.
    $\LL$ is called a \textbf{language of algebras} if $\CR=\emptyset$ and it is a \textbf{language of relational structures} if $\FF=\emptyset$.
\end{definition}

\begin{definition}
    If $\LL$ is a first-order language then a \textbf{first-order structure of type} $\LL$, or an $\LL$-\textbf{structure}, is an ordered pair $\mathbf{A}=\left<A,L\right>$ with $A\neq\emptyset$, where $L$ consists of a family $R$ of \textbf{fundamental relations} $r\mathbf{A}$ on $A$ indexed by $\CR$ and a family $F$ of \textbf{fundamental operations} $f^\mathbf{A}$ on $A$ indexed by $\FF$.
    $A$ is called the \textbf{universe} of $\mathbf{A}$.
    If $\CR=\emptyset$, then $\mathbf{A}$, if $\FF=\emptyset$ then $\mathbf{A}$ is a \textbf{relational structure}
\end{definition}

\begin{definition}
    Let $K$ be a class of structures of type $\LL$, the \textbf{theory} of $K$, written $Th(K)$, is the set of all first-order sentences of type $\LL$ (over some fixed countably infinite set of variables) which are satisfied by $K$.
\end{definition}

\begin{definition}
    Let $\mathbf{A}$ be a structure of type $\LL$ and let $\BBB$ be a structure of type $\LL^*$. 
    Suppose we can find formulas
    \begin{align*}
        & \Delta(x)\\
        & \Phi_f(x_1,\ldots,x_n,y) & \text{for } f\in\FF_n,n\ge 1\\
        & \Phi_r(x_1,\ldots,x_n) & \text{for } r\in\CR_n,n\ge 1
    \end{align*}
    of type $\LL^*$ such that if we let $B_0=\{b\in B:\BBB\models\Delta(b)\}$, then the set
    \[
        \{\left<\left<b_1,\ldots,b_n\right>,b\right>\in B_0^n\times B_0:\BBB\models\Phi_f(b_1,\ldots,b_n,b)\}
    \]
    defines an $n$-ary function $f'$ on $B_0$ for $f\in\FF_n$, $n\ge 1$ and the set
    \[
        \{\left<b_1,\ldots,b_n\right>\in B_0^n:\BBB\models\Phi_r(b_1,\ldots,b_n)\}
    \]
    defines an $n$-ary relation $r'$ on $B_0$ for $r\in\CR_n$, $n\ge 1$ such that by suitably interpreting the constant symbols of $\LL$ in $B_0$ we have a structure $\BBB_0$ of type $\LL$ isomorphic to $\mathbf{A}$.
    Then we say $\mathbf{A}$ can be \textbf{semantically embedded} in $\BBB$, written $\mathbf{A}\xrightarrow[sem]{}\mathbf{B}$.
    If $H$ and $K$ are classes of structures, then $\mathbf{A}\xrightarrow[sem]{}K$ means $\mathbf{A}$ can be semantically embedded in some member of $K$ and $H\xrightarrow[sem]{}K$ means every member of $H$ can be semantically embedded in at least one member of $K$.
\end{definition}

\begin{definition}
    If $K$ is a class of structures of type $\LL$ and $c_1,\ldots,c_n$ are symbols not appearing in $\LL$, then $K(c_1,\ldots,c_n)$ denotes the class of all structures of type $\LL\cup\{c_1,\ldots,c_n\}$, where each $c_i$ is a constant symbol, obtained by taking the members $\BBB$ of $K$ and arbitrarily designating elements $c_1,\ldots,c_n$ in $B$.
\end{definition}

\begin{definition}
    Let $\NN$ be the set of natural numbers and $\left<\NN,+,\cdot,1\right>$ the typical algebra on $\NN$.
\end{definition}

\begin{theorem}[Tarski]
    Given $K$, if for some $n<\omega$ we have $\NN\xrightarrow[sem]{}K(c_1,\ldots,c_n)$, then $Th(K)$ is undecidable.
\end{theorem}

Note that this means the theory of the class of structures $K$ is undecidable if $\NN$ semantically embeds into \textbf{any} member of $K(c_1,\ldots,c_n)$.

\begin{lemma}[Tarski]
    $\NN\xrightarrow[sem]{}\ZZ=\left<\ZZ,+,\cdot,1\right>$
\end{lemma}
\begin{proof}
    Let $\Delta(x)$ be
    \[
        \exists y_1\cdots\exists y_4[x\approx y_1\cdot y_1+\cdots+y_4 \cdot y_4 +1]
    \]
    By a theorem of Langrage, $\ZZ\models\Delta(n)\iff n\in\NN$.
    Let $\Phi_+(x_1,x_2,y)$ be $x_1+x_2\approx y$ and $\Phi_\cdot(x_1,x_2,y)$ be $x_1\cdot x_2\approx y$.
    Then from this, we see $\NN\xrightarrow[sem]{}\ZZ$.
\end{proof}

\begin{theorem}[Tarski]
    The theory of rings is undecidable.
\end{theorem}
\begin{proof}
    If $K$ is the class of rings, $\ZZ\in K$ and so $Th(K)$ is undecidable by $4.7$.
\end{proof}

\begin{lemma}[Tarski]
    $\left<\ZZ,+,\cdot,1\right>\xrightarrow[sem]{}\left<\ZZ,+,{}^2,1\right>$
\end{lemma}
\begin{proof}
    Let $\Delta(x)$ be $x\approx x$, $\Phi_+(x_1,x_2,y)$ be $x_1+x_2\approx y$, $\Phi_\cdot(x_1,x_2,y)$ be $y+y+x_1^2+x_2^2\approx(x_1+x_2)^2$.
    Note $a\cdot b=c\iff 2c+a+b=(a+b)^2$ and so $\Phi_\cdot$ is the multiplication in $\ZZ$.
\end{proof}

\begin{theorem}[Tarski]
    $\left<\ZZ,+,{}^2,1\right>\xrightarrow[sem]{}\left<\ZZ,+,|,1\right>$
\end{theorem}
\begin{proof}
    Let $\Delta(x)$ be $x\approx x$, $\Phi_+(x_1,x_2,y)$ be $x_1+x_2\approx y$ and $\Phi_2(x_1,y)$ be 
    \[\forall z[(x_1+y)|z\iff((x+1|z)\wedge(x_1+1|z)]\wedge\forall u\forall v \forall z[((u+x_1\approx y)\wedge(v+1\approx x_1))\implies(u|z\iff (x_1|z\wedge v|z))]
    \]
    Then $\Phi_2(a,b)$ holds for $a,b\in\ZZ$ if and only if
    \[
            a+b=\pm a(a+1)\\
            b-a=\pm a(a-1)
    \]
    and so if and only if $b=a^2$.
\end{proof}

\begin{lemma}[Tarski]
    Let $Sym(\ZZ)$ be the set of bijections $\ZZ\to\ZZ$ and $\circ$ denote the composition of bijections. 
    Let $\pi$ be the bijection defined by $\pi(a)=a+1$ for each $a\in\ZZ$. Then $\left<\ZZ,+,|,1\right>\xrightarrow[sem]{}\left<Sym(\ZZ),\circ,\pi\right>$.
\end{lemma}
\begin{proof}
    Let $\Delta(x)$ be $x\circ \pi\approx\pi\circ x$, $\Phi_+(x_1,x_2,y)$ be $x_1\circ x_2\approx y$ and $\Phi_|(x_1,x_2)$ be 
    \[
    \forall z(x_1\circ z\approx z\circ x_1\implies x_2\circ z\approx z\circ x_2)
    \]
\end{proof}

\begin{corollary}[Tarski]
    The theory of groups is undecidable.
\end{corollary}
\begin{proof}
    From 4.8, 4.11, 4.12 and the fact that semantic embeddings are transitive, we get $\NN\xrightarrow[sem]{}\left<Sym(\ZZ),\circ,\pi\right>$.
    If $K$ is a class of groups in the language $\{\cdot\}$ then $\left<Sym(\ZZ),\circ,\pi\right>\in K(c_1)$.
    Hence, by $4.7$, $Th(K)$ is undecidable.
\end{proof}

\end{document}