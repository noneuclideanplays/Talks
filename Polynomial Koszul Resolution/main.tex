\documentclass[12pt,a4paper]{article}
\usepackage[english]{babel}
\usepackage[utf8x]{inputenc}
\usepackage[T1]{fontenc}
\usepackage{listings}
\usepackage{amsfonts}
\usepackage{amssymb}
\usepackage{amsmath,amsthm}
\usepackage{mathtools,leftindex,tensor,mhchem}
\usepackage{graphicx}
\usepackage{geometry}
\usepackage{layout}
\usepackage{tikz}
\usepackage{bbm}
\usepackage{tikz-cd}
\usepackage{cite}
\usetikzlibrary{positioning}
\usepackage{latexsym,amsxtra,amscd,ifthen, amsmath,color,url}
\newcommand{\comp}{\overline{\phantom{A}}}
\setlength{\tabcolsep}{12pt}

\newtheorem{theorem}{Theorem}[section]
\newtheorem{lemma}[theorem]{Lemma}
\newtheorem{claim}[theorem]{Claim}
\newtheorem{proposition}[theorem]{Proposition}
\newtheorem{corollary}[theorem]{Corollary}
\newtheorem{fact}[theorem]{Fact}
\newtheorem{example}[theorem]{Example}
\newtheorem{notation}[theorem]{Notation}
\newtheorem{observation}[theorem]{Observation}
\newtheorem{conjecture}[theorem]{Conjecture}
\newtheorem{remark}[theorem]{Remark}
\newtheorem{definition}[theorem]{Definition}
\newcommand\NN{\mathbb{N}} 
\newcommand\RR{\mathbb{R}}
\newcommand\CR{\mathcal{R}}
\newcommand\ZZ{\mathbb{Z}}
\newcommand\AL{\mathcal{A}}
\newcommand\MM{\mathcal{M}}
\newcommand\BB{\mathcal{B}}
\newcommand\BBB{\mathfrak{B}}
\newcommand\CC{\mathcal{C}}
\newcommand\CCC{\mathfrak{C}}
\newcommand\DD{\mathcal{D}}
\newcommand\EE{\mathbb{E}}
\newcommand\FF{\mathcal{F}}
\newcommand\GG{\mathcal{G}}
\newcommand\HH{\mathcal{H}}
\newcommand\JJ{\mathcal{J}}
\newcommand\PP{\mathcal{P}}
\newcommand\QQ{\mathbb{Q}}
\newcommand\CS{\mathcal{S}}
\newcommand\TT{\mathcal{T}}
\newcommand\BT{\mathbb{T}}
\newcommand\OO{\mathcal{O}}
\newcommand\UU{\mathcal{U}}
\newcommand\VV{\mathcal{V}}
\newcommand{\kk}{\Bbbk}
\newcommand\PPT{\mathcal{P}^{*}}
\newcommand\CST{\Sigma^{*}}
\newcommand\RX{\mathsf{X}}
\newcommand\1{_{(1)}}
\newcommand\2{_{(2)}}

%\voffset=-1truein		% LaTeX has too much space at page top
%\addtolength{\textheight}{0.3truein}
%\addtolength{\textheight}{\topmargin}
%\addtolength{\topmargin}{-\topmargin}
%\textwidth  6.0in		% LaTeX article default 360pt=4.98''
%\oddsidemargin 0pt	% \oddsidemargin  .35in   % default is 21.0 pt
%\evensidemargin 0pt	% \evensidemargin .35in   % default is 59.0 pt
%\marginparwidth=0pt
% decrease margins on sides and top/bottom
\geometry{
bottom=25mm,
top=35mm,
left=20mm,
right=20mm
}
\voffset=-0.5truein

\title{Koszul Resolution of a Polynomial Ring}
\date{}

\begin{document}

\maketitle

\section{Koszul Resolution}

Let $\kk$ be a field, $A=\kk[x_1,\ldots,x_n]$, and $V=\kk x_1\oplus\cdots\oplus\kk x_n$.
Fix a basis of \\$A^e=\kk[x_1,\ldots,x_n]\otimes\kk[x_1,\ldots,x_n]$ as a $\kk$-vector space
\[
\{x_1^{i_1}\ldots x_n^{i_n}\otimes x_1^{j_1}\ldots x_n^{j_n}=\vec{x}^{\,\vec{i}}\otimes\vec{x}^{\,\vec{j}}\;\vert\; \vec{i},\vec{j}\in\NN^n\}.
\]
Also fix a basis of $\left(A^e\right)^n$ as an $A^e$-module
\[
\left\{e_i=\begin{bmatrix}0\\\vdots\\1\\\vdots\\0\end{bmatrix}\;\Bigg\vert\; 0\le i\le n, \;1\text{\ in ith row}\right\}.
\]
Finally, fix a basis of $V$ as $\kk$-vector space
\[
\{x_1,\ldots,x_n\}.
\]
The sequence $(x_1\otimes 1-1\otimes x_1,\ldots,x_n\otimes1-1\otimes x_n)$ is clearly regular in $A^e$, so we get a free resolution from the Koszul complex
\[
  0\to A^e\to\bigwedge^{n-1} \left(A^e\right)^n\to\cdots\to\bigwedge^2\left(A^e\right)^n\to\left(A^e\right)^n\to A^e\to 0
\]
of $A^e/(x_1\otimes1-1\otimes x_1,\ldots,x_n\otimes1-1\otimes x_n)\cong A$.
The differentials are given by
\begin{align*}
 \partial_k:\bigwedge^k \left(A^e\right)^k&\to\bigwedge^{k-1}\left(A^e\right)^n\\
  \vec{x}^{\,\vec{i}}\otimes\vec{x}^{\,\vec{j}}e_{m_0}\wedge\cdots\wedge e_{m_{k-1}}&\mapsto\sum_{t=0}^{k-1} (-1)^t\vec{x}^{\,\vec{i}}\otimes\vec{x}^{\,\vec{j}}(x_{m_t}\otimes1-1\otimes x_{m_t})e_{m_0}\wedge\cdots\wedge\widehat{e_{m_t}}\wedge\cdots\wedge e_{m_{k-1}}
\end{align*}
where $\widehat{e_{m_t}}$ means ommitting that term.
Note: $\bigwedge^{k}\left(A^e\right)^n$ is being taken over $A^e$, and so we are identifying $\bigwedge^n\left(A^e\right)^n\cong A^e$ and $\wedge^1\left(A^e\right)^n\cong \left(A^e\right)^n$.

\section{Isomorphic Resolution}

There is an isomorphism of $A^e$-modules 
\begin{align*}
\bigwedge^k\left(A^e\right)^n&\cong A^e\otimes\bigwedge^k V  \\
 \vec{x}^{\,\vec{i}}\otimes\vec{x}^{\,\vec{j}}e_{m_0}\wedge\cdots\wedge e_{m_{k-1}}&\mapsto \vec{x}^{\,\vec{i}}\otimes\vec{x}^{\,\vec{j}}\otimes x_{m_1}\wedge\cdots\wedge x_{m_{k-1}}.
\end{align*}
 Note that $\bigwedge^k V$ is being taken over $\kk$.
This isomorphism induces a chain map isomorphism from the Koszul resolution to the resolution
\[
0\to A^e\to A^e\otimes \bigwedge^{n-1}V\to\cdots\to A^e\otimes\bigwedge^2 V\to A^e\otimes V\to A^e\to 0  
\]
where we identify $A^e\otimes \bigwedge^n V\cong A^e$, $A^e\otimes\bigwedge^1 V\cong A^e\otimes V$.
The induced differentials of this resolution are given by
\begin{align*}
  \partial'_k:A^e\otimes\bigwedge^k V&\to A^e\otimes\bigwedge^{k-1}V\\
  \vec{x}^{\,\vec{i}}\otimes\vec{x}^{\,\vec{j}}\otimes x_{m_0}\wedge\cdots\wedge x_{m_{k-1}}&\mapsto \sum_{t=0}^{k-1}(-1)^t\vec{x}^{\,\vec{i}}\otimes\vec{x}^{\,\vec{j}}(x_{m_t}\otimes1-1\otimes x_{m_t})\otimes x_{m_0}\wedge\cdots\wedge\widehat{x_{m_t}}\wedge\cdots\wedge x_{m_{k-1}}.
\end{align*}
Henceforth, this resolution will be identified as the Koszul resolution of $A$.

\section{Calculating Cohomology}

Recall that as $A$ is a $\kk$-vector space, and so a free $\kk$-module, $HH^k(A)=Ext_{A^e}^k(A,A)$.
This can be computed as the $k$-th homology group of the complex given by from applying the functor $Hom_{A^e}(-,A)$ to the Koszul resolution.
This complex is
\[
0\to Hom_{A^e}(A^e,A)\to Hom_{A^e}(A^e\otimes V,A)\to\cdots\to Hom_{A^e}(A^e\otimes\bigwedge^{n-1}V,A)\to Hom_{A^e}(A^e,A)\to 0  
\]
with differentials
\[
  d_k:Hom_{A^e}(A^e\otimes\bigwedge^k V, A)\to Hom_{A^e}(A^e\bigwedge^{k-1}V,A)
\]
\[
d_k(g)(\vec{x}^{\,\vec{i}}\otimes\vec{x}^{\,\vec{j}}\otimes x_{m_0}\wedge\cdots\wedge x_{m_k})= \sum_{t=0}^{k}(-1)^t\vec{x}^{\,\vec{i}}\otimes\vec{x}^{\,\vec{j}}(x_{m_t}\otimes1-1\otimes x_{m_t})g(1\otimes 1\otimes x_{m_1}\wedge\cdots\wedge\widehat{x_{m_t}}\wedge\cdots\wedge x_{m_k}).
\]
Note that we are using the fact that $g$ is $A^e$-linear to factor out $\vec{x}^{\,\vec{i}}\otimes\vec{x}^{\,\vec{j}}(x_{m_t}\otimes1-1\otimes x_{m_t})$.
As well, $g(1\otimes 1\otimes x_{m_1}\wedge\cdots\wedge\widehat{x_{m_t}}\wedge\cdots\wedge x_{m_k})\in A=\kk[x_1,\ldots,x_n]$, so we can write it as $p(x_1,\ldots,x_n)$.
But 
\[
  (x_{m_t}\otimes1-1\otimes x_{m_t})\cdot p(x_1,\ldots,x_n)=x_{m_t}p(x_1,\ldots,x_n)-p(x_1,\ldots,x_n)x_{m_t}=0
\]
for all $t$ and any $p(x_1,\ldots,x_n)\in A$.
Hence, $d_k(g)=0$ for all $g\in Hom_{A^e}(A^e\otimes\bigwedge^k V, A)$ and for all $k$, so every differential is the $0$ map.
Then, the homology groups are $Ext_{A^e}^k(A,A)=Hom_{A^e}(A^e\otimes\bigwedge^k V, A)$ for all $n\le k\le 0$.
\\\\
There is an isomorphism of $A^e$-modules 
\begin{align*}
  A\otimes \bigwedge^k V&\cong Hom_{A^e}(A^e\otimes\bigwedge^k V, A)\\
  \vec{x}^{\,\vec{i}}\otimes x_{m_0}\wedge\cdots\wedge x_{m_{k-1}}&\mapsto
  (1\otimes 1\otimes x_{m_0}\wedge\cdots\wedge x_{m_{k-1}}\mapsto \vec{x}^{\,\vec{i}}).
\end{align*}
Therefore, $HH^k(\kk[x_1,\ldots,x_n])=\kk[x_1,\ldots,x_n]\otimes\bigwedge^k\left(\kk x_1\oplus\cdots\oplus\kk x_n\right)$.

\section{Calculating Homology}

As with the cohomology, since $A$ is a free $\kk$-module, $HH_k(A)=Tor_k^{A^e}(A,A)$.
This can be calculated as the $k$-th homology group of the complex given by applying the functor $-\otimes_{A^e}A$ to the Koszul resolution.
This complex is 
\[
  0\to A^e\otimes_{A^e}A\to A^e\otimes\bigwedge^{n-1}V\otimes_{A^e} A\to\cdots\to A^e\otimes V\otimes_{A^e}A\to A^e\otimes_{A^e}A\to 0
\]
\[
d_k':A^e\otimes\bigwedge^k V\otimes_{A^e}A\to A^e\otimes\bigwedge^{k-1}V\otimes_{A^e}A
\]
\begin{align*}
&d_k'(\vec{x}^{\,\vec{i}}\otimes\vec{x}^{\,\vec{j}}\otimes x_{m_{0}}\wedge\cdots\wedge x_{m_{k-1}}\otimes_{A^e}\vec{x}^{\,\vec{l}})=\\
&\sum_{t=0}^{k-1}(-1)^t\vec{x}^{\,\vec{i}}\otimes\vec{x}^{\,\vec{j}}(x_{m_t}\otimes1-1\otimes x_{m_t})\otimes x_{m_0}\wedge\cdots\wedge\widehat{x_{m_t}}\wedge\cdots\wedge x_{m_{k-1}}\otimes_{A^e}\vec{x}^{\,\vec{l}}.
\end{align*}
But as the functor $-\otimes_{A^e} A$ is tensoring over $A^e$, we can rewrite this as
\begin{align*}
  \sum_{t=0}^{k-1}(-1)^t\vec{x}^{\,\vec{i}}\otimes\vec{x}^{\,\vec{j}}(x_{m_t}\otimes1-1\otimes x_{m_t})\otimes x_{m_0}\wedge\cdots\wedge\widehat{x_{m_t}}\wedge\cdots\wedge x_{m_{k-1}}\otimes_{A^e}\vec{x}^{\,\vec{l}}=\\
  \sum_{t=0}^{k-1}(-1)^t\vec{x}^{\,\vec{i}}\otimes\vec{x}^{\,\vec{j}}\otimes x_{m_0}\wedge\cdots\wedge\widehat{x_{m_t}}\wedge\cdots\wedge x_{m_{k-1}}\otimes_{A^e}(x_{m_t}\otimes1-1\otimes x_{m_t})\vec{x}^{\,\vec{l}}=\\
  0.
\end{align*}
So again, all of the differentials are $0$, and hence $HH_k(A)=A^e\otimes\bigwedge^k V\otimes_{A^e}A$.
\\\\
As with the cohomology, there is an isomorphism of $A^e$-modules
\begin{align*}
  A^e\otimes\bigwedge^k V\otimes_{A^e}A&\cong A\otimes\bigwedge^k V\\
  \vec{x}^{\,\vec{i}}\otimes\vec{x}^{\,\vec{j}}\otimes x_{m_0}\wedge\cdots\wedge x_{m_{k-1}}\otimes_{A^e}\vec{x}^{\,\vec{l}}
  &\mapsto \vec{x}^{\,\vec{i}}\vec{x}^{\,\vec{l}}\vec{x}^{\,\vec{j}}\otimes x_{m_0}\wedge\cdots\wedge x_{m_{k-1}}.
\end{align*}
Therefore, $HH_k(\kk[x_1,\ldots,x_n])=\kk[x_1,\ldots,x_n]\otimes\bigwedge^k \left(\kk x_1\oplus\cdots\oplus\kk x_n\right)$.

\end{document}