\documentclass[12pt,a4paper]{article}
\usepackage[english]{babel}
\usepackage[utf8x]{inputenc}
\usepackage[T1]{fontenc}
\usepackage{listings}
\usepackage{amsfonts}
\usepackage{amssymb}
\usepackage{amsmath,amsthm}
\usepackage{mathtools}
\usepackage{graphicx}
\usepackage{geometry}
\usepackage{layout}
\usepackage{tikz}
\usepackage{tikz-cd}
\usepackage{bussproofs}
\usepackage{stackengine}[2013-09-11]
\usetikzlibrary{positioning}
\usepackage{etoolbox,refcount,multicol}
\newcommand{\comp}{\overline{\phantom{A}}}
\setlength{\tabcolsep}{12pt}

\usepackage{etex} % to fix "no room for new dimen" error. Voodoo.
\usepackage{array}
\usepackage{varwidth}
\usepackage{bussproofs}
\usepackage{epigraph}
\usepackage{stmaryrd}
\usepackage{mathdots}
\usepackage{amsmath, amscd, amssymb, mathrsfs, accents, amsfonts,amsthm}
\usepackage[all]{xy}
\usepackage{mathtools} % for bra-ket
\usepackage{tikz}
\usetikzlibrary{calc}
\newcommand{\tagarray}{\mbox{}\refstepcounter{equation}$(\theequation)$}

\newenvironment{mathprooftree}
  {\varwidth{.9\textwidth}\centering\leavevmode}
  {\DisplayProof\endvarwidth}

\def\drawbang{\draw[color=teal!50, line width=2pt]}
\def\drawprom{\draw[color=gray, line width=3pt]}
\def\bluenode{\node[circle,draw=blue!50,fill=blue!20]}
\def\mapnode{\node[circle,draw=black,fill=black,inner sep=0.5mm]}
\def\whitenode{\node[circle,draw=blue!50,fill=blue!5]}
\def\dernode{\node[circle,draw=black,fill=white]}
\definecolor{Myblue}{rgb}{0,0,0.6}
\usepackage[a4paper,colorlinks,citecolor=Myblue,linkcolor=Myblue,urlcolor=Myblue,pdfpagemode=None]{hyperref}

\newtheorem{theorem}{Theorem}[section]
\newtheorem{lemma}[theorem]{Lemma}
\newtheorem{claim}[theorem]{Claim}
\newtheorem{proposition}[theorem]{Proposition}
\newtheorem{corollary}[theorem]{Corollary}
\newtheorem{fact}[theorem]{Fact}
\newtheorem{example}[theorem]{Example}
\newtheorem{notation}[theorem]{Notation}
\newtheorem{observation}[theorem]{Observation}
\newtheorem{conjecture}[theorem]{Conjecture}
\newtheorem{remark}[theorem]{Remark}
\newtheorem{definition}[theorem]{Definition}
\newcommand\NN{\mathbb{N}} 
\newcommand\RR{\mathbb{R}}
\newcommand\CR{\mathcal{R}}
\newcommand\ZZ{\mathbb{Z}}
\newcommand\AL{\mathcal{A}}
\newcommand\MM{\mathcal{M}}
\newcommand\BB{\mathcal{B}}
\newcommand\BBB{\mathfrak{B}}
\newcommand\CC{\mathcal{C}}
\newcommand\CCC{\mathfrak{C}}
\newcommand\BC{\mathbb{C}}
\newcommand\DD{\mathcal{D}}
\newcommand\FF{\mathcal{F}}
\newcommand\GG{\mathcal{G}}
\newcommand\HH{\mathcal{H}}
\newcommand\JJ{\mathcal{J}}
\newcommand\PP{\mathcal{P}}
\newcommand\QQ{\mathbb{Q}}
\newcommand\CS{\mathcal{S}}
\newcommand\TT{\mathcal{T}}
\newcommand\BT{\mathbb{T}}
\newcommand\OO{\mathcal{O}}
\newcommand\UU{\mathcal{U}}
\newcommand\VV{\mathcal{V}}
\newcommand\PPT{\mathcal{P}^{*}}
\newcommand\CST{\Sigma^{*}}
\newcommand\RX{\mathsf{X}}
\newcommand\XX{\mathcal{X}}

%multi-col itemize
\newcounter{countitems}
\newcounter{nextitemizecount}
\newcommand{\setupcountitems}{%
  \stepcounter{nextitemizecount}%
  \setcounter{countitems}{0}%
  \preto\item{\stepcounter{countitems}}%
}
\makeatletter
\newcommand{\computecountitems}{%
  \edef\@currentlabel{\number\c@countitems}%
  \label{countitems@\number\numexpr\value{nextitemizecount}-1\relax}%
}
\newcommand{\nextitemizecount}{%
  \getrefnumber{countitems@\number\c@nextitemizecount}%
}
\newcommand{\previtemizecount}{%
  \getrefnumber{countitems@\number\numexpr\value{nextitemizecount}-1\relax}%
}
\makeatother    
\newenvironment{AutoMultiColItemize}{%
\ifnumcomp{\nextitemizecount}{>}{3}{\begin{multicols}{2}}{}%
\setupcountitems\begin{itemize}}%
{\end{itemize}%
\unskip\computecountitems\ifnumcomp{\previtemizecount}{>}{3}{\end{multicols}}{}}

%\voffset=-1truein		% LaTeX has too much space at page top
\addtolength{\textheight}{0.3truein}
\addtolength{\textheight}{\topmargin}
\addtolength{\topmargin}{-\topmargin}
\textwidth  6.0in		% LaTeX article default 360pt=4.98''
%\oddsidemargin 0pt	% \oddsidemargin  .35in   % default is 21.0 pt
%\evensidemargin 0pt	% \evensidemargin .35in   % default is 59.0 pt
%\marginparwidth=0pt
% decrease margins on sides and top/bottom
\geometry{
bottom=20mm,
margin=20mm
}
\title{Not Another Categorization}
\date{}
\begin{document}%\layout
	%\MakeScribeTop
	

	
\maketitle

\begin{abstract}
Linear logic is a system encompassing classical logic that is keeps track of premises.
We will describe the structure of linear logic and its deductive system.
Intuitionistc linear logic then has a natural categorical form in a very well-known category.
We will describe this category model of intuitionistic linear logic.
\end{abstract}

\section{Linear Logic}

\begin{definition}
    Let $Var=\{x_1,x_2,\ldots\}$ be a collection of countably many variables, intuitionistic linear logic (ILL) has two binary connectives, $\multimap$ and $\otimes$, a unary connective, $!$, and a constant $1$.
    Formulas in $ILL$ are defined recursively in the typical way, any variable or constant is a formula and if $A$ and $B$ are formulas so are $A\multimap B$, $A\otimes B$ and $!A$.
\end{definition}

What is fundamentally different about linear logic from classical logic is that it is resource-sensitive.
This means if $A$ proves both $B$ and $C$, we cannot conclude $B$ and $C$ from $A$ as we only have one $A$.
That is where the connective $!$ comes in, $!A$ is an "infinite" resource.
Due to this the contraction and weakening rules are modified.
One could argue this method is closer the natural language, for example if "If I have 2 dollars then I can buy a soda" and "If I have 2 dollars then I can buy a candy bar" are true, then having 2 dollars would mean you could buy a candy bar and a soda in classical logic, but not in reality.
So you must pay attention to exactly how many of the resource "dollars" you have.
The tensor product $\otimes$ acts similar to conjunction, but $A\otimes B$ does not prove $A$ or $B$ and $A$ does not prove $A\otimes A$.


The deduction rules of ILL are as follows:

% Axiom rule
\begin{center}
\begin{tabular}{ >{\centering}m{6cm} >{\centering}m{6cm} >{\centering}m{2cm}}
\AxiomC{}
\LeftLabel{(Axiom): }
\UnaryInfC{$A \vdash A$}
\DisplayProof
\end{tabular}


% Exchange rule ======

\begin{tabular}{ >{\centering}m{6cm} >{\centering}m{6cm} >{\centering}m{2cm}}
\AxiomC{$\Gamma, A, B, \Delta \vdash C$}
\LeftLabel{(Exchange): }
\UnaryInfC{$\Gamma, B, A, \Delta \vdash C$}
\DisplayProof
\end{tabular}
\end{center}

% Cut rule ======
\begin{center}
\begin{tabular}{ >{\centering}m{6cm} >{\centering}m{6cm} >{\centering}m{2cm}}
\AxiomC{$\Gamma \vdash A$} 
\AxiomC{$\Delta', A, \Delta \vdash B$}
\LeftLabel{(Cut): }
\RightLabel{\scriptsize cut}
\BinaryInfC{$\Delta',\Gamma, \Delta \vdash B$}
\DisplayProof
\end{tabular}
\end{center}


% Right tensor ======
\begin{center}
\begin{tabular}{ >{\centering}m{6cm} >{\centering}m{6cm} >{\centering}m{2cm}}
\AxiomC{$\Gamma \vdash A$} \AxiomC{$\Delta \vdash B$}
\LeftLabel{(Right $\otimes$): }
\RightLabel{\scriptsize $\otimes$-$R$}
\BinaryInfC{$\Gamma, \Delta \vdash A \otimes B$}
\DisplayProof
\end{tabular}
\end{center}

% Left tensor ======
\begin{center}
\begin{tabular}{ >{\centering}m{6cm} >{\centering}m{6cm} >{\centering}m{2cm}}
\AxiomC{$\Gamma, A, B, \Delta \vdash C$}
\LeftLabel{(Left $\otimes$): }
\RightLabel{\scriptsize $\otimes$-$L$}
\UnaryInfC{$\Gamma, A \otimes B, \Delta \vdash C$}
\DisplayProof
\end{tabular}
\end{center}

% Right Hom ======
\begin{center}
\begin{tabular}{ >{\centering}m{6cm} >{\centering}m{6cm} >{\centering}m{2cm}}
\AxiomC{$A, \Gamma \vdash B$}
\LeftLabel{(Right $\multimap$): }
\RightLabel{\scriptsize $\multimap R$ }
\UnaryInfC{$\Gamma \vdash A \multimap B$}
\DisplayProof
\end{tabular}
\end{center}

% Left Hom ======
\begin{center}
\begin{tabular}{ >{\centering}m{6cm} >{\centering}m{6cm} >{\centering}m{2cm}}
\AxiomC{$\Gamma \vdash A$} \AxiomC{$\Delta', B, \Delta \vdash C$}
\LeftLabel{(Left $\multimap$): }
\RightLabel{\scriptsize $\multimap L$ }
\BinaryInfC{$\Delta', \Gamma, A \multimap B, \Delta \vdash C$}
\DisplayProof
\end{tabular}
\end{center}

% Promotion ======
\begin{center}
\begin{tabular}{ >{\centering}m{6cm} >{\centering}m{6cm} >{\centering}m{2cm}}
\AxiomC{$! \Gamma \vdash A$}
\LeftLabel{(Promotion): }
\RightLabel{\scriptsize prom}
\UnaryInfC{$!\Gamma \vdash !A$}
\DisplayProof
\end{tabular}
\end{center}

% Dereliction ======
\begin{center}
\begin{tabular}{ >{\centering}m{6cm} >{\centering}m{6cm} >{\centering}m{2cm}}
\AxiomC{$\Gamma, A, \Delta \vdash B$}
\LeftLabel{(Dereliction): }
\RightLabel{\scriptsize der}
\UnaryInfC{$\Gamma, !A, \Delta \vdash B$}
\DisplayProof
\end{tabular}
\end{center}

% Contraction ======
\begin{center}
\begin{tabular}{ >{\centering}m{6cm} >{\centering}m{6cm} >{\centering}m{2cm}}
\AxiomC{$\Gamma, !A, !A, \Delta \vdash B$}
\LeftLabel{(Contraction): }
\RightLabel{\scriptsize ctr}
\UnaryInfC{$\Gamma, !A, \Delta \vdash B$}
\DisplayProof
\end{tabular}
\end{center}

% Weakening ======
\begin{center}
\begin{tabular}{ >{\centering}m{6cm} >{\centering}m{6cm} >{\centering}m{2cm}}
\AxiomC{$\Gamma, \Delta \vdash B$}
\LeftLabel{(Weakening): }
\RightLabel{\scriptsize weak}
\UnaryInfC{$\Gamma, !A, \Delta \vdash B$}
\DisplayProof
\end{tabular}
\end{center}

% Left 1 ======
\begin{center}
\begin{tabular}{ >{\centering}m{6cm} >{\centering}m{6cm} >{\centering}m{2cm}}
\AxiomC{$\Gamma, \Delta \vdash A$}
\LeftLabel{(Left $1$): }
\RightLabel{\scriptsize $1$-$L$}
\UnaryInfC{$\Gamma, 1, \Delta \vdash A$}
\DisplayProof
\end{tabular}
\end{center}

% Right 1 ======
\begin{center}
\begin{tabular}{ >{\centering}m{6cm} >{\centering}m{6cm} >{\centering}m{2cm}}
\AxiomC{}
\LeftLabel{(Right $1$): }
\RightLabel{\scriptsize $1$-$R$}
\UnaryInfC{$\vdash 1$}
\DisplayProof
\end{tabular}
\end{center}

Towards the goal of considering semantics of ILL, we consider a categorization.
Each formula $A$ is assigned an object $[[A]]$ and each proof $\Gamma\vdash A$ assigned a morphism $[[\Gamma]]\to[[A]]$.
This assignment is module cut-elimination, so if two proofs are equivalent under cut-elimination then they are assigned the same morphism.
From the deduction rules, one can see the category formed must be a symmetric monoidal closed category with a comonad.


\section{Category}
\begin{definition}
    A symmetric monoidal category is a tuple $(\CC,\otimes,a,L,R,1,\tau)$ where $\otimes:Obj(\CC)\to Obj(\CC)$ is a functor, $a,L,R$ are families of morphisms describing associatvity and that $A\otimes1=A=1\otimes A$, and $\tau$ is a family of isomorphisms $\tau_{A,B}:A\otimes B\to B\otimes A$ so that $\tau_{B,A}\circ\tau_{A,B}=id_{A\otimes B}$.
\end{definition}

To say the category is closed means a morphism $A\otimes B\to C$ can be viewed as a morphism $X\to Hom(Y,Z)$ where $Hom(Y,Z)$ is now treated as an object in the category.
This is the same as the concept of currying, changing a function of two variables into an operation on one variable returning a function in the other variable.\\

\begin{definition}
    If $V$ is a vector space, then there is a corresponding cofree cocommutative coalgebra that is universal and we will denote it as $!V$.
\end{definition}

\begin{definition}
    A linear category is a symmetric monoidal closed category with a comonad defined by the $!V$.
\end{definition}

\begin{definition}
    We define $[[A]]$ as follows:
    \begin{enumerate}
        \item a variable $x$ is assigned a finite dimensional vector space $[[x]]$
        \item $[[1]]=k$
        \item $[[A\otimes B]]=[[A]]\otimes[[B]]$
        \item $[[A\multimap B]]=[[A]]\multimap [[B]]:=Hom(A,B)$
        \item $[[!A]]=![[A]]$
        \item if $\Gamma=A_1,\ldots,A_n$, then $[[\Gamma]]=[[A_1]]\otimes\cdots\otimes[[A_n]]$, and if $\Gamma=\emptyset$ then $[[\Gamma]]=k$
    \end{enumerate}
\end{definition}

Now what do proofs look like in the category?

\begin{definition}
    If $\pi$ is a proof of $\Gamma\vdash B$, then $[[\pi]]$ is a linear map $[[\Gamma]]\to[[B]]$.
\end{definition}

\begin{itemize}
\item The diagram for the axiom rule depicts the identity of $[[A]]$.
\item The diagram for the exchange rule uses the symmetry $[[B]] \otimes [[A]] \to [[A]] \otimes [[B]]$.
\item The diagram for the cut rule depicts the composition of the two inputs.
\item The right tensor rule depicts the tensor product of the two given morphisms, while the left tensor rule depicts the identity, viewed as a morphism between two strands labelled $[[A]]$ and $[[B]]$ and a single strand labelled $[[A]] \otimes [[B]]$.
\item The diagram for the right $\multimap$ rule denotes the adjoint of the input morphism while the left $\multimap$ rule uses the composition map.
\item The diagram for the promotion rule depicts the lifting of the input to a morphism of coalgebras.
\item The diagram for the dereliction rule depicts the composition of the input with the universal map out of the coalgebra ${!} V$.
\end{itemize}

\end{document}