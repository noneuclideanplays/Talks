\documentclass[12pt,a4paper]{article}
\usepackage[english]{babel}
\usepackage[utf8x]{inputenc}
\usepackage[T1]{fontenc}
\usepackage{listings}
\usepackage{amsfonts}
\usepackage{amssymb}
\usepackage{amsmath,amsthm}
\usepackage{mathtools}
\usepackage{graphicx}
\usepackage{geometry}
\usepackage{layout}
\usepackage{tikz}
\usetikzlibrary{positioning}
\newcommand{\comp}{\overline{\phantom{A}}}
\setlength{\tabcolsep}{12pt}

\newtheorem{theorem}{Theorem}[section]
\newtheorem{lemma}[theorem]{Lemma}
\newtheorem{claim}[theorem]{Claim}
\newtheorem{proposition}[theorem]{Proposition}
\newtheorem{corollary}[theorem]{Corollary}
\newtheorem{fact}[theorem]{Fact}
\newtheorem{example}[theorem]{Example}
\newtheorem{notation}[theorem]{Notation}
\newtheorem{observation}[theorem]{Observation}
\newtheorem{conjecture}[theorem]{Conjecture}
\newtheorem{remark}[theorem]{Remark}
\newtheorem{definition}[theorem]{Definition}
\newcommand\NN{\mathbb{N}} 
\newcommand\RR{\mathbb{R}}
\newcommand\CR{\mathcal{R}}
\newcommand\ZZ{\mathbb{Z}}
\newcommand\AL{\mathcal{A}}
\newcommand\MM{\mathcal{M}}
\newcommand\BB{\mathcal{B}}
\newcommand\BBB{\mathfrak{B}}
\newcommand\CC{\mathcal{C}}
\newcommand\CCC{\mathfrak{C}}
\newcommand\BC{\mathbb{C}}
\newcommand\DD{\mathcal{D}}
\newcommand\FF{\mathcal{F}}
\newcommand\GG{\mathcal{G}}
\newcommand\HH{\mathcal{H}}
\newcommand\JJ{\mathcal{J}}
\newcommand\PP{\mathcal{P}}
\newcommand\QQ{\mathbb{Q}}
\newcommand\CS{\mathcal{S}}
\newcommand\TT{\mathcal{T}}
\newcommand\BT{\mathbb{T}}
\newcommand\OO{\mathcal{O}}
\newcommand\UU{\mathcal{U}}
\newcommand\VV{\mathcal{V}}
\newcommand\PPT{\mathcal{P}^{*}}
\newcommand\CST{\Sigma^{*}}
\newcommand\RX{\mathsf{X}}

%\voffset=-1truein		% LaTeX has too much space at page top
\addtolength{\textheight}{0.3truein}
\addtolength{\textheight}{\topmargin}
\addtolength{\topmargin}{-\topmargin}
\textwidth  6.0in		% LaTeX article default 360pt=4.98''
%\oddsidemargin 0pt	% \oddsidemargin  .35in   % default is 21.0 pt
%\evensidemargin 0pt	% \evensidemargin .35in   % default is 59.0 pt
%\marginparwidth=0pt
% decrease margins on sides and top/bottom
\geometry{
bottom=20mm,
margin=20mm
}
\title{Determining Decidability Algebraically}
\date{}
\begin{document}%\layout
	%\MakeScribeTop
	

	
\maketitle

\begin{abstract}
In this talk we will contextualize common algebraic notions in Universal Algebra and provide examples of translations into Universal Algebra.
Then we will discuss the logical notions of Universal Algebra and how to use these to perform Model Theoretic proofs.
Finally we will show how to use Universal Algebra to show the undecidiability of certain theories, in particular the theory of groups.
\end{abstract}
\section{Algebras}

\begin{definition}
Let $A\neq\emptyset$ be a set and $n\in\mathbb{Z}$. An \textbf{$n$-ary operation/function} on $A$ is a function $f:A^n\to A$
\end{definition}

\begin{definition}
A \textbf{language/type} of algebras is a set $\mathcal{F}$ of function symbols such that each function has an arity
\end{definition}

\begin{definition}
Let $\mathcal{F}$ be a language of algebras, then an \textbf{algebra} $\mathbb{A}$ \textbf{of type } $\mathcal{F}$ is an ordered pair $\left<A,F\right>$ where $A$ is a non-empty set and $F$ is a family of finitary operations on $A$ indexed by $\mathcal{F}$ such that corresponding to each $n$-ary $f\in\mathcal{F}$ there is an $n$-ary operation $f^\mathbb{A}$ on $A$
\end{definition}

We call $A$ the \textbf{universe} of $\mathbb{A}=\left<A,F\right>$.

\begin{example}
A \textbf{groupoid} or \textbf{magma} is an algebra $\mathbb{G}=\left<G,\cdot\right>$ with only a single binary operation.\\
A \textbf{group} is an algebra $\mathbb{G}=\left<G,\cdot,{}^{-1},1\right>$ such that $x\cdot(y\cdot z)\approx (x\cdot y)\cdot z$, $x\cdot 1\approx 1\cdot x\approx x$ and $x\cdot x^{-1}\approx x^{-1}\cdot x\approx 1$.\\
A \textbf{Boolean algebra} is an algebra $\mathbb{B}=\left<B,\vee,\wedge,',0,1\right>$ such that $\left<B,\vee,\wedge\right>$ is a distributive lattice, $x\wedge 0\approx 0$, $x\vee 1\approx 1$, $x\wedge x'\approx 0$ and $x\vee x'\approx 1$.\\
\end{example}

\begin{definition}
We call $\mathbb{B}$ a \textbf{subalgebra} of $\mathbb{A}$ if $B\subseteq A$ and $f^\mathbb{B}=f^\mathbb{A}\upharpoonright B$.\\
We call $B$ a \textbf{subuniverse} of $\mathbb{A}$ if $B\subseteq A$ and 
\[
f(b_1,\ldots,b_n)\in B  \;\;\;\;\; \text{for all } n\text{-ary } f \text{ and } b_i\in B
\]
\end{definition}

\begin{definition}
Let $\mathbb{A}$ be an algebra and $X\subseteq A$, the \textbf{subuniverse generated by } $X$ is $Sg(X)=\bigcap\{B:X\subseteq B, \text{ B a subuniverse of } \mathbb{A}\}$
\end{definition}
Remark: We can define a lattice $\mathbb{L}_{Sg}$ of subuniverses of $\mathbb{A}$.

\begin{theorem}
If $\mathbb{A}$ is an algebra and $X\subseteq A$, then $|Sg(X)|\le |X|+|\mathcal{F}|+\omega$
\end{theorem}

\section{Algebraic Notions}

\begin{definition}
Let $\mathbb{A}$ be an algebra of type $\mathcal{F}$ and let $\theta$ be an equivalence relation on $A$, $\theta\in Eq(A)$. Then we call $\theta$ a \textbf{congruence} on $\mathbb{A}$ if $\theta$ satisfies the \textbf{compatibility property}: For each $n$-ary function symbol $f\in\mathcal{F}$ and elements $a_i,b_i\in A$, if $a_i\theta b_i$ holds for $1\le i\le n$, then 
\[
f^\mathbb{A}(a_1,\ldots,a_n)\theta f^\mathbb{A}(b_1,\ldots,b_n)
\]
holds.
\end{definition}

In a sense, this means the equivalence relation is compatible with the structure of the group, algebraic operations on equivalent elements yields equivalent elements.

\begin{example}
$\mathbb{Z}=\left<\mathbb{Z},+\right>$ is a group with the congruence relation $\mod n$. If $a\equiv b \mod n$ and $c\equiv d\mod n$ then $a+c\equiv b+d\mod n$.
\end{example}

\begin{definition} The set of all congruence relations on an algebra $\mathbb{A}$ is denoted $\mathbf{Con}\mathbb{A}$. Let $\theta\in\mathbf{Con}\mathbb{A}$, the \textbf{quotient algebra} of $\mathbb{A}$ by $\theta$, written $\mathbb{A}/\theta$, is the algebra whose universe is $A/\theta$ and whose operations satisfy 
\[
f^{\mathbb{A}/\theta}(a_1/\theta,\ldots,a_n/\theta)=f^{\mathbb{A}}(a_1,\ldots,a_n)/\theta
\]
\end{definition}

Note: $\mathbb{A}/\theta$ is the set of equivalence classes of elements of $A$ modulo $\theta$, and so I will usually write $[a]_\theta$ instead of $a/\theta$ as that is the more familiar notation for equivalence classes.

\begin{example}
Let $G$ be a group and $N$ a normal subgroup, we can define the binary relation $ \left<a,b\right> \in\theta\iff ab^{-1}\in N$.
This relation is then a congruence on $G$ as it is clearly an equivalence relation and $[a_1]_\theta*_{G/\theta}[a_2]_\theta=[a_1*_{G}a_2]_\theta$, the compatibility property on group multiplication. This congruence in particular is special as $[1]_\theta=N$.\\
Similarly, given any $\theta\in$Con($G$), $[1]_\theta$ is the universe of a normal subgroup of $G$.
\end{example}

\begin{definition}
Let $\mathbb{A}$ and $\mathbb{B}$ be two algebras of the same type $\mathcal{F}$, a mapping $\alpha:A\to B$ is called a \textbf{homomorphism} from $\mathbb{A}$ to $\mathbb{B}$ if 
\[
\alpha f^\mathbb{A}(a_1,\ldots,a_n)=f^\mathbb{B}(\alpha a_1,\ldots,\alpha a_n)
\]\\
The \textbf{kernel} of $\alpha$ is a congruence on $\mathbb{A}$ defined by $ker(\alpha)=\{ \left<a,b\right> \in A^2:\alpha(a)=\alpha(b)\}$.\\
\end{definition}

\begin{definition}
Let $\mathbb{A}$ be an algebra and $\theta\in$Con($\mathbb{A}$), the \textbf{natural map} $\upsilon_\theta:A\to A/\theta$ is defined by $\upsilon_\theta(a)=[a]_\theta$ and is always an onto homomorphism.
\end{definition}

\begin{theorem}[First Isomorphism Theorem]
Suppose $\alpha:A\to B$ is a homomorphism onto $B$, then there is an isomorphism $\beta$ from $\mathbb{A}/\ker{\alpha}$ to $B$ defined by $\alpha=\beta\circ\upsilon$.
\end{theorem}

\begin{definition}
Let $\mathbb{A}$ be an algebra and $\theta,\phi\in$Con($\mathbb{A}$) with $\theta\subseteq\phi$, then we define $\phi/\theta=\{\left<[a]_\theta,[b]_\theta\right>\in(A/\theta)^2: \left<a,b\right> \in\phi\}$.
\end{definition}

\begin{lemma}
$\phi/\theta$ is a congruence on $\mathbb{A}/\theta$.
\end{lemma}

\begin{theorem}[Second Isomorphism Theorem]
If $\phi,\theta\in$Con($\mathbb{A}$) and $\theta\subseteq\phi$, then the map $(A/\theta)/(\phi/\theta)\to A/\phi$ defined by $\alpha(([[a]_\theta]_{\phi/\theta}))=[a]_\phi$ is an isomorphism.
\end{theorem}

\begin{definition}
Suppose $B\subseteq A$ and $\theta\in$Con($\mathbb{A}$). Let $B^\theta=\{a\in A:B\cap[a]_\theta\neq\emptyset\}$ and $\mathbb{B}^\theta$ be the subalgebra of $\mathbb{A}$ generated by $B^\theta$. We also define  $\theta\upharpoonright B$ to be the restriction of $\theta$ to $B$, $\theta\cap B^2$.
\end{definition}

\begin{theorem}[Third Isomorphism Theorem]
If $\mathbb{B}$ is a subalgebra of $\mathbb{A}$ and $\theta\in$Con($\mathbb{A}$), then $\mathbb{B}/\theta\upharpoonright B\cong\mathbb{B}^\theta/\theta\upharpoonright B^\theta$.
\end{theorem}

\begin{definition}
We call an algebra $\mathbb{A}$ \textbf{simple} if Con($\mathbb{A}$)$=\{\{\left<a,a\right>\},A^2\}$.\\
We call a congruence $\theta$ \textbf{maximal} if there is no larger, in the $\subseteq$ sense, congruence that is not $A^2$.
\end{definition}

\begin{definition}
A \textbf{term} in an algebra $\mathbb{A}$ is a recursively defined string of symbols from $A$ strung together with function symbols from $\mathcal{F}$.
\end{definition}

\begin{example}
Given a group $G=\left<G,*,1\right>$ where $G=\{x,y\}$, some terms include $1$, $x$, $x*y$, $1*(x*y)$, some examples of non-terms are $1*$ and $xy$.
\end{example}

\begin{definition}
Let $\mathbb{A}$ be an algebra of type $\mathcal{F}$. The \textbf{center} of $\mathbb{A}$ is the binary relation $Z(\mathbb{A})$ defined by $\left<a,b\right>\in Z(\mathbb{A})$ if and only if for every term $p(x,y_1,\ldots,y_n)$ and every $c_1,\ldots,c_n,d_1,\ldots,d_n\in A$, $p(a,c_1,\ldots,c_n)=p(a,d_1,\ldots,d_n)\iff p(b,c_1,\ldots,c_n)=p(b,d_1,\ldots,d_n)$
\end{definition}

\begin{example}
Let $G=\left<G,*,{}^{-1},1\right>$ be a group, if $ \left<a,b\right> \in Z(G)$ then consider the term $p(x,y_1,y_2)=y_1*x*y_2$ and let $c\in G$. We then have
$$p(a,a^{-1},c)=a^{-1}*a*c=c=c*a*a^{-1}=p(a,c,a^{-1})$$
Hence, 
\[
p(b,a^{-1},c)=p(b,c,a^{-1})\implies a^{-1}*b*c=c*b*a^{-1} 
\]
by the definition of $ \left<a,b\right> \in Z(G)$.
Letting $c=1$ we get $a^{-1}*b=b*a^{-1}$, and so $A$ and $b$ are elements of the usual notion of the subgroup $Z(G)$.
One can also work backwards
\end{example}

\section{Latin Squares}

\begin{definition}
    A \textbf{quasigroup} is an algebra $\left<Q,/,\cdot,\backslash,1\right>$ satisfying the identities
    \begin{align*}
        (Q1)\; x\backslash (x\cdot y)\approx y\;\;\;\;\; & (x\cdot y)/y\approx x\\
        (Q2)\; x\cdot(x\backslash y)\approx y \;\;\;\;\;& (x/y)\cdot y\approx x
    \end{align*}
\end{definition}

\begin{theorem}
A finite groupoid $\mathbb{A}$ is a quasigroup if and only if every element of $A$ appears exactly once in each row and column of the Cayley table.
\end{theorem}

\begin{definition}
    A \textbf{Latin square of order n} is an $nx\times n$ matrix $(a_{ij}$ of elements from an $n$ element set $A$ such that each member of $A$ occurs exactly once in each row and column of the matrix.
\end{definition}

Clearly there is a one-to-one correspondence between Latin squares and quasigroups.

\begin{definition}
    If $(a_{ij})$ and $(b_{ij})$ are two Latin squares of order $n$ with entries from $A$ with the property that for each $ \left<a,b\right> \in A\times A$ there is exactly one index $ij$ such that $\\ \left<a,b\right> =\left<a_{ij},b_{ij}\right>$, then we say that $(a_{ij})$ and $(b_{ij})$ are orthogonal.
\end{definition}

In other words, every ordered pair from $A^2$ appears as an ordered pair of squares with the same index from the Latin squares.

\begin{example}
    \begin{tabular}{|c|c|c|}
        \hline
        a & b & c\\
        \hline
        b & c & a\\
        \hline
        c & a & b\\
        \hline
    \end{tabular}\;\;\;
    \begin{tabular}{|c|c|c|}
        \hline
        a & b & c\\
        \hline
        c & a & b\\
        \hline
        b & c & a\\
        \hline
    \end{tabular}
    are two orthogonal Latin squares with entries from $A=\{a,b,c\}$.
\end{example}

\begin{conjecture}[Euler]
If $n\equiv 2 \mod 4$ then there are no orthogonal Latin squares of order $n$.
\end{conjecture}

Euler was only ever able to show this for the special case $n=2$, and then in $1900$, Tarry proved it for $n=6$.
But eventually it was proven that this conjecture was false, in 1960 Bose, Parker and Shrikhande proved that $n=2$ and $6$ are the only cases where this is true.
We will show a construction orthogonal Latin Squares of order $54$ with the methods of Universal algebra.

\begin{definition}
    A \textbf{pair of orthogonal Latin squares} is an algebra $\left<A,/,\cdot,\backslash,/^*,\circ,\backslash^*,*_1,*_r\right>$ with eight binary operations such that
    \begin{align*}
        (i)&\; \left<A,/,\cdot,\backslash\right> \text{\;\; is a quasigroup}\\
        (ii)&\; \left<A,/^*,\circ,\backslash^*\right> \text{\;\; is a quasigroup}\\
        (iii)&\; *_1(x\cdot y,x\circ y) \approx x\\
        (iv)&\; *_r(x\cdot y,x\circ y)\approx y
    \end{align*}
    We say the \textbf{order} of the algebra is $|A|$.
    Define POLS to be the \textbf{variety of pairs of orthogonal Latin squares}.
\end{definition}

\begin{definition}
    An algebra $\left<A,F\right>$ is a \textbf{binary algebra} if each of the fundamental operations is binary. A binary algebra is called \textbf{idempotent} if $f(x,x)\approx x$ holds for each function symbol $f$.
    In turn, a variety $V$ of algebras is \textbf{binary idempotent} if the members of $V$ are binary idempotent algebras and $V$ can be defined by identities involving at most two variables.
\end{definition}

\begin{definition}
    A $2$-design is a tuple $\left<B,B_1,\ldots,B_k\right>$ where
    \begin{align*}
        (i)\;\; & B \text{ is a finite set}\\
        (ii) \;\;& \text{each } B_i \text{ is a subset of } B \text{ called a \textbf{block}}\\
        (iii) \;\;& |B_i|\ge 2 \text{ for all } i\\
        (iv) \;\;& \text{each two-element subset of } B \text{ is contained in exactly one block}
    \end{align*}
\end{definition}

\begin{lemma}
Let $V$ be a binary idempotent variety and let $\left<B,B_1,\ldots,B_k\right>$ be a $2$-design.
Let $n=|B|, n_i=|B_i|$, if $V$ has members of size $n_i$, $1\le i\le k$, then $V$ has a member of size $n$.
\end{lemma}

\begin{proof}
Let $\mathbb{A}_i\in V$ with $|A_i|=n_i$. 
We can assume $A_i=B_i$. 
Then, for each binary function symbol $f$ in the type of $V$ we can find a binary function $f^B$ on $B$ such that when we restrict $f^B$ to $B_i$ it agrees with $f^{\mathbb{A}_i}$ (we essentially let $f^B$ be the union of the $f^{\mathbb{A}_i}$). 
As $V$ can be defined by two variable identities $p(x,y)\approx q(x,y)$ which hold on each $\mathbb{A}_i$, it follows that we have constructed an algebra $\mathbb{B}$ in $V$ with $|B|=n$.
\end{proof}

\begin{lemma}
If $q$ is a prime power and $q\ge 4$, then there is an idempotent member of POLS of size $q$.
In particular, there are members of sizes $5, 7$ and $8$.
\end{lemma}
\begin{proof}
Let $\mathbb{K}$ be a field of order $q$, let $e_1$ and $e_2$ be two distinct elements of $K\setminus\{0,1\}$, and define two binary operations $*_1$ and $*_2$ on $K$ by 
\[
a*_i b=e_i\cdot a+(1-e_i)\cdot b
\]
The Cayley tables of $\left<K,*_1\right>$ and $\left<K,*_2\right>$ give rise to an idempotent pair of orthogonal Latin squares.
\end{proof}

Now some simple projective geometry we will take for granted.
Given a finite field $F$ with cardinality $n+1$, we form the projective plane $\PP_n$ of order $n$ by letting the points be the subsets of $F^3$ of the form $a+U$ where $a\in F^3$ and $U$ is a one-dimensional subspace of $F^3$ as a vector space and letting the lines be the subsets of $F^3$ of the form $a+V$ where $a\in F^3$ and $V$ a two-dimensional subspace of $F^3$.
Then every line of $\PP_n$ contains $n+1$ points and every point is contained in $n+1$ lines.
There are in total $n^2+n+1$ points and lines.
Finally, any two distinct points belong to exactly one line and any two distinct lines meet in exactly one point.

\begin{lemma}
There is a $2$-design $\left<B,B_1,\ldots,B_k\right>$ with $|B|=54$ and $|B_i|\in\{5,7,8\}$ for $1\le i\le k$.
\end{lemma}
\begin{proof}
Let $\pi$ be the projective plane of order $7$.
This has $57$ points and each line contains $8$ points.
Choose three points on one line and remove them.
Let $B$ be the set of the remaining $54$ points and let $B_i$ be the sets obtained by intersecting the lines of $\pi$ with $B$.
Then $\left<B,B_1,\ldots,B_k\right>$ is easily seen to be a $2$-design since each pair of points from $B$ lies on a unique line of $\pi$ and $|B_i|\in\{5,7,8\}$.
\end{proof}

\begin{theorem}
There is an idempotent pair of orthogonal Latin squares of order $54$.
\end{theorem}
\begin{proof}
Lemmas $3.12, 3.13 $ and $3.14$ together prove this.
\end{proof}

Note that $54 \equiv 2 \mod 4$ and so we have shown Euler's original conjecture false.

\end{document}