\makeatletter
\def\moverlay{\mathpalette\mov@rlay}
\def\mov@rlay#1#2{\leavevmode\vtop{%
   \baselineskip\z@skip \lineskiplimit-\maxdimen
   \ialign{\hfil$\m@th#1##$\hfil\cr#2\crcr}}}
\newcommand{\charfusion}[3][\mathord]{
    #1{\ifx#1\mathop\vphantom{#2}\fi
        \mathpalette\mov@rlay{#2\cr#3}
      }
    \ifx#1\mathop\expandafter\displaylimits\fi}
\makeatother

\newcommand{\abs}[1]{\left\vert#1\right\vert}
\newcommand{\cupdot}{\charfusion[\mathbin]{\cup}{\cdot}}
\newcommand{\bigcupdot}{\charfusion[\mathop]{\bigcup}{\cdot}}

\documentclass{amsart}

\newtheorem{theorem}{Theorem}[section]
\newtheorem{lemma}[theorem]{Lemma}
\newtheorem{corollary}[theorem]{Corollary}
\newtheorem{definition}[theorem]{Definition}
\newtheorem{observation}[theorem]{Observation}
\newtheorem{construction}[theorem]{Construction}
\newtheorem{case}{Case}

\usepackage{amsmath}
\usepackage{amssymb}
\usepackage[top=1.5in, left=1in, right=1in, bottom=1in]{geometry}

\usepackage[utf8]{inputenc}
\usepackage{graphicx}

\title{Dihedral Game}
\author{Logan Crone, Lior Fishman, Mckenzie Fontenot, Naomi Krawzik, Brandon Mather, Erin Raign, Julie Thompson}
\date{January 2020}

\begin{document}

\maketitle

\section{Introduction}

\section{Definitions}

\begin{definition}
Let $G$ be a finite group. We play the following game: \medskip \\
 \begin{tabular}{c|ccc}
 \hline
  Player 1 & $g_0 \in G$ & & $g_2 \in G \backslash (cl\{g_0\} \cup cl\{g_1\})$ \\
  Player 2 & & $g_1 \in G \backslash cl\{g_0\}$ 
  \end{tabular} \medskip \\
   where $g_n \in G\backslash \bigcup_{i<n} cl\{g_i\}$ and $g_n \not = e$. Define $p_n = g_0 g_1 \cdot \cdot \cdot g_n$.\\
    If both players run out of moves, it is a draw. Otherwise, the first player to make $p_n = e$ wins.
\end{definition}

\begin{theorem}
The game played on $D_n$ has the following properties 
\begin{enumerate}
    \item If $n$ is odd, both players can force a draw
    \item If $n\equiv 2$ mod $4$ then Player $1$ wins
    \item If $n \equiv 0$ mod $4$, then we have
    \begin{enumerate}
     \item Player 1 wins if $n \equiv 0$ mod $8$
        \item Player 2 wins if $n \equiv 4$ mod $8$
     \end{enumerate}
\end{enumerate}
\end{theorem}

\begin{proof}
(1) Suppose $n$ is odd, then the conjugacy classes of $D_n$ are $\{e\}, \{\mu \rho^k:k\le n\}$ and $\frac{n-1}{2}$ of $\{\rho^k,\rho^{n-k}\}$.
If we wish for the product of the moves to equal $e$, then neither player can play an element from the single $\mu$ class as the inverse cannot be played after.
Thus both players can force a draw by playing an element from the $\mu$ class, so neither player wins.\\

(2) Suppose $n\equiv 2$ mod $4$.
\begin{case}
Player $2$ plays $\rho^i$ for $i \in \{1,...,n-1\}$.
\end{case}


\end{proof}

\end{document}