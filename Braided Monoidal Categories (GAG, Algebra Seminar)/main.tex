\documentclass[12pt,a4paper]{article}
\usepackage[english]{babel}
\usepackage[utf8x]{inputenc}
\usepackage[T1]{fontenc}
\usepackage{listings}
\usepackage{amsfonts}
\usepackage{amssymb}
\usepackage{amsmath,amsthm}
\usepackage{mathtools}
\usepackage{graphicx}
\usepackage{geometry}
\usepackage{layout}
\usepackage{tikz}
\usepackage{tikz-cd}
\usepackage{bussproofs}
\usepackage{stackengine}[2013-09-11]
\usetikzlibrary{positioning}
\usepackage{etoolbox,refcount,multicol}
\newcommand{\comp}{\overline{\phantom{A}}}
\setlength{\tabcolsep}{12pt}

\newtheorem{theorem}{Theorem}[section]
\newtheorem{lemma}[theorem]{Lemma}
\newtheorem{claim}[theorem]{Claim}
\newtheorem{proposition}[theorem]{Proposition}
\newtheorem{corollary}[theorem]{Corollary}
\newtheorem{fact}[theorem]{Fact}
\newtheorem{example}[theorem]{Example}
\newtheorem{notation}[theorem]{Notation}
\newtheorem{observation}[theorem]{Observation}
\newtheorem{conjecture}[theorem]{Conjecture}
\newtheorem{remark}[theorem]{Remark}
\newtheorem{definition}[theorem]{Definition}
\newcommand\NN{\mathbb{N}} 
\newcommand\RR{\mathbb{R}}
\newcommand\CR{\mathcal{R}}
\newcommand\ZZ{\mathbb{Z}}
\newcommand\AL{\mathcal{A}}
\newcommand\MM{\mathcal{M}}
\newcommand\BB{\mathcal{B}}
\newcommand\BBB{\mathfrak{B}}
\newcommand\CC{\mathcal{C}}
\newcommand\CCC{\mathfrak{C}}
\newcommand\BC{\mathbb{C}}
\newcommand\DD{\mathcal{D}}
\newcommand\FF{\mathcal{F}}
\newcommand\GG{\mathcal{G}}
\newcommand\HH{\mathcal{H}}
\newcommand\JJ{\mathcal{J}}
\newcommand\PP{\mathcal{P}}
\newcommand\QQ{\mathbb{Q}}
\newcommand\CS{\mathcal{S}}
\newcommand\TT{\mathcal{T}}
\newcommand\BT{\mathbb{T}}
\newcommand\OO{\mathcal{O}}
\newcommand\UU{\mathcal{U}}
\newcommand\VV{\mathcal{V}}
\newcommand\PPT{\mathcal{P}^{*}}
\newcommand\CST{\Sigma^{*}}
\newcommand\RX{\mathsf{X}}
\newcommand\XX{\mathcal{X}}

%multi-col itemize
\newcounter{countitems}
\newcounter{nextitemizecount}
\newcommand{\setupcountitems}{%
  \stepcounter{nextitemizecount}%
  \setcounter{countitems}{0}%
  \preto\item{\stepcounter{countitems}}%
}
\makeatletter
\newcommand{\computecountitems}{%
  \edef\@currentlabel{\number\c@countitems}%
  \label{countitems@\number\numexpr\value{nextitemizecount}-1\relax}%
}
\newcommand{\nextitemizecount}{%
  \getrefnumber{countitems@\number\c@nextitemizecount}%
}
\newcommand{\previtemizecount}{%
  \getrefnumber{countitems@\number\numexpr\value{nextitemizecount}-1\relax}%
}
\makeatother    
\newenvironment{AutoMultiColItemize}{%
\ifnumcomp{\nextitemizecount}{>}{3}{\begin{multicols}{2}}{}%
\setupcountitems\begin{itemize}}%
{\end{itemize}%
\unskip\computecountitems\ifnumcomp{\previtemizecount}{>}{3}{\end{multicols}}{}}

%\voffset=-1truein		% LaTeX has too much space at page top
\addtolength{\textheight}{0.3truein}
\addtolength{\textheight}{\topmargin}
\addtolength{\topmargin}{-\topmargin}
\textwidth  6.0in		% LaTeX article default 360pt=4.98''
%\oddsidemargin 0pt	% \oddsidemargin  .35in   % default is 21.0 pt
%\evensidemargin 0pt	% \evensidemargin .35in   % default is 59.0 pt
%\marginparwidth=0pt
% decrease margins on sides and top/bottom
\geometry{
bottom=20mm,
margin=20mm
}
\title{Braided Monoidal Categories and ?}
\date{}
\begin{document}%\layout
	%\MakeScribeTop
	

	
\maketitle

\begin{abstract}
We will define Braided Monoidal Categories and describe their relation to the Braid Group.
\end{abstract}

\section{Braids}

\begin{definition}
    
\end{definition}


\section{Categories}

As we are attempting, in a way, to form Category Theory as a foundation of math, we will define things in a metamathematical sense.

\begin{definition}
A \textit{category} $\mathcal{ C}$ is a collection of objects, $\text{Ob}\mathcal{C}$, and arrows (morphisms) and operations:
\begin{itemize}
\begin{AutoMultiColItemize}
    \item Domain: assigns each arrow an object, $\text{dom}f=a$
    \item Codomain: assigns each arrow an object, $\text{cod}f=b$
    \item Identity: assigns each object an arrow, $\text{id}_a=1_a:a\to a$
    \item Composition: assigns to each pair of arrows $\left<f,g\right>$ with $\text{dom}f=\text{cod}g$ an arrow $g\circ f$
\end{AutoMultiColItemize}
\end{itemize}
The operations are given certain restrictions:
\begin{itemize}
    \item Associativity: For objects $a,b,c$ and $d$ and arrows $f:a\to b$, $g:b\to c$, $k:c\to d$, we have the equality $k\circ(g\circ f)=(k\circ g)\circ f$. Represented as a commuting diagaram:\\
    \begin{center}
    \begin{tikzcd}[sep = 20mm]
       a \arrow[rr,"k\circ(g\circ f)=(k\circ g)\circ f"]
         \arrow[d,"f"]
         \arrow[drr,"g\circ f" near start]
         & & d \\ b \arrow[urr,"k\circ g" near start]
                    \arrow[rr,"g"] & &  c \arrow[u,"k"]
    \end{tikzcd}
    \end{center}
    \item Unit Law: For all arrows $f:a\to b$ and $g:b\to c$, composition with identity arrows $1_b$ gives $1_b\circ f=f$ and $g\circ 1_b=g$. Represented as a commuting diagram:\\
    \begin{center}
    \begin{tikzcd}[sep = 15 mm]
        a \arrow[r,"f"] \arrow[dr,"f"] & b \arrow[d,"1_b"] \arrow[dr,"g"] & \\
                                       & b \arrow[r,"g"]                  & c
    \end{tikzcd}
    \end{center}
\end{itemize}
\end{definition}

\begin{example}
The category of sets, \textbf{Set}, has as its objects all sets and its arrows all functions between sets. 
Note that the collection of objects is never given to be a set, interpreted in set theory it can be a proper class. 
In the category of sets, we have identity function $1_X:X\to X$ defined by $x\mapsto x$. 
We also have composition, if $f:X\to Y$ and $g: Y\to Z$ then $g\circ f:X\to Z$ is defined by $g(f(x))$, and this composition is well-known to be associative.
\end{example}

\begin{example}
    The collection of groups and group homomorphisms forms the category $\textbf{Grp}$.
    There also is the category \textbf{Ab} of abelian groups and group homomorphisms.
\end{example}

\begin{example}
    \begin{tikzcd}
        {\bullet} \arrow[loop]
    \end{tikzcd}
    \begin{tikzcd}
        {\bullet} \ar[loop left] \ar[r] & {\bullet} \ar[loop right]
    \end{tikzcd}
Are both vacuously categories.
\end{example}

\begin{definition}
    A \textit{functor} is a "morphism of categories". 
    Specifically, if we have to categories $\mathcal{C}$ and $\mathcal{B}$, a functor $T:\CC\to\BB$ consists of two functions: the \textit{object function} $T$ assigns each object $c\in\CC$ an object $Tc\in\BB$, and the \textit{arrow function} $T$ assigning each arrow $f:c\to c'$ of $\CC$ an arrow $Tf:Tc\to Tc'$ of $\BB$.
    The functor must respect the structure of the categories, $T(1_c)=1_{Tc}$ and $T(g\circ f)=Tg\circ Tf$.
\end{definition}

\begin{example}
    Let $\textbf{Set}$ be the category of sets, a simple example of a functor is the power set $\PP:\textbf{Set}\to\textbf{Set}$.
    Each set $X$ is assigned a set $\PP(X)$ and each arrow $f:X\to Y$ gets assigned a map $\PP f:\PP(X)\to\PP(Y)$.
\end{example}

\begin{example}
    A group, $G$ can be considered a category in itself with only one object, the group, and an arrow for each element of the group.
    A functor from $G$ to \textbf{Set} is then a group action of $G$ on a set.
    Similarly, if a functor maps $G$ to $\textbf{Vect}_K$, the category of vector spaces over $K$, is a linear representation of $G$.
\end{example}

\begin{definition}
    Given two functors $S,T:\CC\to\BB$, a \textit{natural transformation}, "morphism of functors", $\tau:S\to T$ is a function which assigns to each object $c$ of $\CC$ an arrow $\tau c:Sc\to Tc$ of $\BB$ in such a way that every arrow $f:c\to c'$ in $\CC$ yields a commuting diagram\\
    \begin{center}
    \begin{tikzcd}[sep=15mm]
        c\arrow[d,"f"] & Sc \arrow[r,"\tau c"] \arrow[d,"Sf"] & Tc \arrow[d,"Tf"]\\
        c'             & Sc'\arrow[r,"\tau c'"]               & Tc'
    \end{tikzcd}
    \end{center}
\end{definition}

 The natural transformation can be thought of as translating the image under $S$ to the image under $T$.
 While one of the harder concepts to intuit, this is one of the most important notions in Category Theory.

 \begin{example}
     Let $\textbf{Grp}$ be the category of groups and $\textbf{CRng}$ be the category of commutative rings.
     There is a functor $GL_n:\textbf{CRng}\to\textbf{Grp}$ defined by taking a commutative ring $K$ to the group of invertible $n\times n$ matrices $GL_n(K)$.
     We also have a functor ${}^*:\textbf{CRng}\to\textbf{Grp}$ defined by sending a commutative ring $K$ to its group of units $K^*$.
     Between these functors is a natural transformation, $det:GL_n\to{}^*$.
     \begin{center}
         \begin{tikzcd}[sep=15mm]
            GL_n K \arrow[r,"det_K"] \arrow[d,"GL_n f"] & K^* \arrow[d,"f^*"]\\
            GL_n K' \arrow[r,"det_{K'}"]                & K^{'*}
         \end{tikzcd}
     \end{center}   
 \end{example}



\end{document}